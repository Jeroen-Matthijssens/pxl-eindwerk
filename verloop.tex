\section {Project verloop}

\subsection {Tools en project setup}

\paragraph {Docker} docker is awesome want is applicatie onafhankelijk, en draaid native
op os, geen voledige vm nodig, wel nog altijd afgedwongen limiten op processor gebruik,
ram, schijf ... .

Zelf heb ik de deployment niet gedaan. Daar is docker mss een beetje onwennig om mee te
werken, (geen idee)

docker is was nog heel jong, de management tools stonden nog in de kinder shoenen
(kubernetes, google) was toen pas op github gezet in an as is toestand. Dus niet echt
bruikbaar, is ondertussen mss genoeg gevalueerd om bruikbaar te zijn.

Er waren een aantal andere manier om er mee te werken shipyard, ... maar de meeste die ik
heb gezien waren nogal clumsy.


\subsection {Beperkingen van Sencha Extjs}

\paragraph {} Sencha Extjs is een javascript
framework\footnote{https://www.sencha.com/products/extjs/\#overview} om crossplatform web
applications te maken voor de desktop, tablet en smartphone. Ze willen gebruik kunnen maken
met HTML5 features, maar toch backwards compatible blijven met oudere browsers.

\paragraph {} De layout wordt berekend door hun layout engine die in javascript de hoogte
en breedte berekent van alle componenten, en dan deze waarden als \lstinline{height} en
\lstinline{width} zet op de html elementen. Dit betekend dat op het grote aantal dom-elementen
in je html css stylen worden overschreven. Om deze positionering goed te laten
werken plaatst Sencha de visuele componenten op de pagina in een extra div element dat door het
framework gebruikt kan worden om deze berekeningen op te baseren.

Dit houdt ook in dat een deel van de layout en styling hoort te gebeuren in de view
'classes' die je in Sencha maakt. Dit klinkt natuurlijk logisch, maar elke andere visuele
configuratie dient te gebruiren in afzonderlijke css bestanden. Dus niet alle informatie
die de layout van je pagina beschrijft staat bij elkaar.

Normaal doe je de styling van je paginss in css, zodat je styling en content van elkaar
gescheiden blijven. Maar omdat Sencha absoluut zijn layout engine wil gebruiken MOET je de
positionering van je componenten wel in code doen.

\paragraph {} Sencha bied een object georiënteerde manier van programeren aan. Javascript
heeft objecten, maar zijn niet de objecten die je zou verwachten als je een object
georiënteerde taal gewoon bent (zoals java, ruby, smalltalk, ...). Javascript is een
prototyped taal, geen object georiëenteerde taal. Als je probeert OO semantiek in
javascript te introduceren stuit je op problemen.

Een goed voorbeeld is het gebruik van \lstinline{this}. In javascript heeft dit een andere
betekenis dan bijvoorbeeld in java. Er is geen enkele manier om hier rond te werken. In
javascript kan je functies meegeven als argumenten. Dit komt vaak voor onder de vorm van
een callback. Als de functie klaar is met de berekeningen wordt de callback opgeroepen. Op
dat moment is \lstinline{this} in de callback niet meer de \lstinline{this} die je had op
het moment dat je de andere functien oproept.

\begin{lstlisting}[language=ownjavascript]
Ext.define ('SomeClass', {
	first: function first () {
		Ext.Ajax.request ({
			...
			callback: this.handleResponse
		});
	},
	handleResponse: function handleResponse () {
		...
	}
});
\end{lstlisting}

De \lstinline{handleResponse ()} method wordt wel megegeven, maar je kan niet meer aan de
waarden van de \lstinline{SomeClass}. Er zijn een aantal oplossingen voor dat probleem. De
eerste is een variable introduceren om de waarden van \lstinline{this} vast te houden.

\begin{lstlisting}[language=ownjavascript]
var me = this;
Ext.Ajax.request ({ callback: function () { me.handleResponse (); }, ... });
\end{lstlisting}

Een andere optie is om je 'scope' mee te geven aan \lstinline{request ()}. Dit is een
manier van werken die op veel plaatsen in Sencha ondersteunt wordt.

\begin{lstlisting}[language=ownjavascript]
Ext.Ajax.request ({ callback: this.handleResponse, scope: this, ... });
\end{lstlisting}

Je kan ook javascript features gebruiken \lstinline{callback: me.handleRespnose.bind (this)}

\begin{lstlisting}[language=ownjavascript]
Ext.Ajax.request ({ callback: this.handleResponse.bind (this), ... });
\end{lstlisting}


\paragraph {} In Sencha code zie je op veel plaatsen steevast \lstinline{var me = this;}
staan. Alle functies worden dat op het \lstinline{me} object opgeroepen, op die manier
merk je het verschil tussen Sencha als oo layer en javascript minder, maar het is geen
oplossing. Je hebt nog altijd dezelfde problemen. \lstinline{callback: me.handleResponse}
gaat nog altijd een error veroorzaken.

\paragraph {} Het heeft geen zin om bepaalde functies als \lstinline{static} te markeren,
in javascript bestaat dit niet. Je kan de naam of het woord er wel rond plaatsen, maar dat
zorgt er niet voor dat dit zich als static methods of properties gaat gedragen.

Door te doen alsof je op een OO manier kan programmeren in javascript
hou je echte begrip van de taal waar je echt mee werkt tegen. In Sencha lijkt men er
vanuit te gaan dat je OO nodig hebt om code te structureren, dit is zeker niet het geval.
Is het dan niet beter om je code te structureren met technieken die toepasselijk zijn in
de taal waarmee je werkt?


\paragraph {} build tools komen niet overeen met wat je zou verwachten voor een javascript
applicatie. De build tools zijn gebazeerd op ant. Terwijl de rest van de wereld

Dus ook geen echte versie voor packages die je zelf maakt.


\paragraph {} Sencha wil je zo veel mogelijk scheiden van html css, en echte javascript.
(bvb Sencha architect).

\paragraph {} Het mechanisme dat Sencha gebruikt om te bepalen welke listeners op
viewmodels moeten uitgevoerd worden werkt enkel goed voor het ingebouwde framework. Je kan
dit gebruiken door een aantal configuraties. Als je dit op andere manieren wilt gebruiken
werket dit goed zolang je wilt luisteren naar veranderingen van primiteve waarden.

\begin{lstlisting}[language=ownjavascript]
var model = Ext.create ('Ext.app.ViewModel');
model.bind ('{test}', function onTestChange (newValue, oldvalue) {
	console.log ('got a new value', newValue);
});

model.set ('test', 0); // got a new value 0
model.set ('test', model.get ('test') + 1);
\end{lstlisting}

Dit werkt ook nog goed met diep geneste waarde.

\begin{lstlisting}[language=ownjavascript]
function notified (path) {
	return fuction (newValue, oldValue) {
		console.log (path, 'changed from', oldValue, 'to', newValue);
	}
}

model.bind ('{root.test}', notified ('root.test'));

model.set ('root.test', 0);
// root.test changef from undefined to 0
model.set ('root.test', model.get ('test') + 1);
// root.test changef from 0 to 1

model.get ('root'); // { test: 1 }
\end{lstlisting}

Willen we ook gewaarschuwd worden als er waarden dieper op het object aangepast worden
kunnen we dit met wat configuratie.

\begin{lstlisting}[language=ownjavascript]
model.bind ('{root}', notified ('root'));
model.set ('root.test', 0);
// nothing was notified

model.bind ('{root}', notified ('root'), null, { deep: true});
model.set ('root.test', 0);
// root changef undefined to Object {test: 0}
\end{lstlisting}

Maar dit werkt niet meer als we niet de \lstinline{set ()} method van de viewmodel
gebruiken om de waarden rechtstreeks te veranderen. Sencha zal achter de schremen
controleren of de de nieuwe waarde die het krijgt veranderd is (als de \lstinline{===}
vergelijking \lstinline{false} teruggeeft).

\begin{lstlisting}[language=ownjavascript]
model.bind ('{root.deep}', notified ('root.deep.value'), null, { deep: true });

var deep = model.get ('root.deep');
deep.value = 'someValue';
// nothing is notified
model.set ('root.deep', deep);
// nothing is notified
\end{lstlisting}

In dit geval zou je eenvoudig gewoon de waarde kunnen zetten, maar dit werkt niet meer als
we het object doorgeven aan een andere functie die geen viewmodels aanvaard. Er bestaat
ook geen mechanisme of toch een de listeners op zo een waarde te waarschruwen.

\begin{lstlisting}[language=ownjavascript]
var deep = model.get ('root.deep');
completeDeepValues (deep);
\end{lstlisting}

\paragraph {} Angular js behandeld dit probleem op een heel andere manier. Er worden twee
versien bijgehouden, de echte data op de view model, en de toestand van de data voor user
code wordt uitgevoerd. Op het einde loop Angular over de twee object en controleerd of er
ergens wijzigingen zijn aangebracht. Dit lijkt een performantie nachtmerrie, maar HET
WERKT TENMINSTE!


\paragraph {} erg reactief en dus traag voor nieuwe dingen. (pas NU een versies waar je
code kan delen tussen mobile en desktop applicaties).


\subsection {Template herstructurering}

\paragraph {} Een voorbeeld waar Sencha het lastig maakt is in het templating systeem. Dit
is eigenlijk iets dat intern door het framework gebruikt. Daarbuiten kunnen eindgebruikers
van het framework zelf templates maken, hoewel het afgeraden wordt. Het templating
gedeelte is daardoor niet erg flexibel.

\paragraph {} Toen we de schemas wilde tonen waren er twee mogelijke methoden van aanpak.
We konden de Sencha grid met stores gebruiken, of we konden zelf een template maken die
gebruik maakte van de stores om data op het scherm te tonen. Er zijn een aantal varianten
nodig van die template.

Op de standaard Sencha manier zou je code moeten bijschrijven, of moet je twee keer de
template maken, en eentje licht aanpassen voor het nieuwe gebruik, met andere woorden
duplicatie.


\subsection {Normale Sencha XTemplate werking}

\paragraph {} Sencha templates werken door het gebruik van \lstinline{<tpl />} tags in de
html die je wilt generenen. Deze tags krijgen dan speciale betekenis op basis van de
attributen die er in voorkomen.

Om waarden in het eind resultaat te incorporeren wordt
gebruik gemaakt door uitdrukkingen tussen \lstinline|{ ... }| te zetten. De waarde van de
uitdrukking wordt dan in de html opgenomen. De expressie binnen de accolades wordt
ge\"evalueerd ten opzichten van een speciaal \lstinline{values} object (later hier meer
over).

Het is ook mogelijk on \lstinline|{[ ... ]}|
te gebruiken. In dit geval wordt de uitdrukking niet ge\"evalueerd tenopzichte van het
\lstinline{values} object, maar als gewone jaavscript. Als laatste heb je ook nog de
mogelijk om uitdrukkingen tussen \lstinline|{\% ... \%}| te plaatsen. Dit is javascript
die uitgevoerd wordt, maar niet in de gegeneerde html verschijnt.

\begin{lstlisting}[language=xml]
<tpl for="users">
	<tpl if="active">
		
		{name} {[ values.lastname.toUpperCase () ]}
	</tpl>
</tpl>
\end{lstlisting}

\begin{lstlisting}[language=ownjavascript]
var users = [
	{ name: 'John', lastname: 'Doe', active: false },
	{ name: 'Jane', lastname: 'Doe', active: true }
];
var tpl = Ext.create ('Ext.XTemplate', getAboveString ());
tpl.apply ({ users: users });
// Jane DOE
\end{lstlisting}

\paragraph {} Als je een template hebt kan je deze toepassen op een javascript object. Dit
object telt dan als het \lstinline{values} object voor de template. Als je expressies
gebruikt in \lstinline{<tpl />} attribute wordt dit uitgevoerd binnen de context van het
\lstinline{values} object.

\paragraph {} Als de template een \lstinline{<tpl for="..." />} tegenkomt zal het lopen
over alle waarde, maar binnen in de loop wordt het \lstinline{values} object veranderd
naar het object van de huidige iteratie van de loop. Op dit nieuwe object wordt dan het
voorgaande object bereikbaar via de \lstinline {parent} variable.

\begin{lstlisting}[language=xml]
<tpl for="users">
	{name} is een van de {[ parent.users.length ]} personen.
	{name} is echt een van de {parent.users.length} personen.
</tpl>

<!-- resultaat
John is een van de 2 personen.
John is echt een van de 2 personen.

Jane is een van de 2 personen.
Jane is echt een van de 2 personen.
-->
\end{lstlisting}

\paragraph {} Het parent object heeft zelf geen parent porperty. Willen we nu bijvoorbeeld
een dubbele for lus hebben, dan kunnen we niet meer aan onze oorspronkelijke context.

\begin{lstlisting}[language=ownjavascript]
var users = [
	{ name: 'John', lastname: 'Doe', active: false, phones: [ '012 / 34 56 78', ...] },
	...
];
var tpl = Ext.create ('Ext.XTemplate', getTplString ());
tpl.apply ({ users: users, countryCode: '+ 32' });
\end{lstlisting}

\begin{lstlisting}[language=xml]
<tpl for="users">
	{naam} is bereikbaar op
	<tpl for="phones">
		{parent.parent.countryCode} {.}
	</tpl>
</tpl>

<!-- resultaat
John is bereikbaar op
012 / 34 56 78
...
-->
\end{lstlisting}

\lstinline|{.}| evalueert naar het huidige \lstinline{values} object als een string.
\lstinline|{#}| evalueert naar de index van de huidige loop. Deze index begint te tellen
vanaf 1.

\paragraph {} Je kan de template ook een aantal methods meegeven zodat deze kunnen
gebruikt worden binnen in de template.

\begin{lstlisting}[language=ownjavascript]
var tpl = Ext.create ('Ext.XTemplate', getTplString (), {
	greeting: function greeting () {
		return 'greeting';
	},
	formatNumber: function formatNumber (number) {
		return number.toString ().replace ('.', ',');
	}
});
tpl.apply ({ example: 1.2 });
\end{lstlisting}

\begin{lstlisting}[language=xml]
{[ this.greeting () ]}
{example:this.formatNumber}

<!-- resultaat
greeting
1,2
-->
\end{lstlisting}

\paragraph {} De templating ziet er uit als xml/html, maar dat is het niet

\begin{lstlisting}[language=xml]
<tpl if="condition">
	...
<tpl elseif="condition2">
	...
<tpl alse>
	...
</tpl>
\end{lstlisting}


\subsection {Nieuwe werking}

\paragraph {}Het Meest basic dat de meeste templating engine hebben is een manier om een
andere template te gebruiken door inclusie. Sencha ondersteunt dit niet.

Daarbij kan je templates op twee manieren uitbreiden.

Je zou je kunnen voorstellen dat een
template net het zelfde is als een andere template, op een aantal stukken na. Je template
zou dan verwijzen naar een andere template, en het is die template dat de eigenlijke
structuur bepaalt. Bijvoorbeeld een web pagina, waar altijd een menu vanboven staat samen
met een loge, een footer met contact informatie beneden. Maar elke pagina is een beetje
anders, de eigenlijke inhoud van de pagina veranderd.

Aan de andere kant kan een template hebben die de structuur aangeeft, maar op een aantal
plaatsen verwijst naar andere templates die de stukjes in vullen. Bijvoorbeeld een blog
past waar onderaan een form staat om opmerkingen achter te laten.

\paragraph {} Niets van dit is mogelijk met Sencha templates. Met de nieuwe templates kan
je deze afhankelijkheden in de template zelf specificeren. Je moet dus geen code meer
schrijven om andere templates toe te includen. Om de templates aan elkaar te kopelen
bestaat er iets als een template store. Dit is een verzameling van templates die onderling
naar elkaar kunnen verwijzen door middel van een naam. Je kan deze templates gewoon in een
xml file bij elkaar zetten.

\begin{lstlisting}[language=xml]
<? xml version="1.0" encoding="UTF-8" ?>
<templates>

	<template name="base">
		first line
		<tpl block="second-line" />
		lastline
	</template>

	<template name="main">
		<tpl extend="base" />

		<tpl block="second-line">
			middel line
		</tpl>
	</template>

</templates>
\end{lstlisting}

\begin{lstlisting}[language=xml]
<? xml version="1.0" encoding="UTF-8" ?>
<templates>

	<template name="main">
		...
		<tpl include="footer" />
	</template>

	<template name="footer">
		...
	</template>

</templates>
\end{lstlisting}

De nieuwe functionalitiet is toegankelijk door gebruik te maken van een aantal
\lstinline{<tpl />} tags. We tonen later hoe deze tags samenwerken en voor meer
flexibiliteit zorgen.

\begin{enumerate}
	\item \lstinline{<tpl include="" />}
	\item \lstinline{<tpl extend="" />}
	\item \lstinline{<tpl block="" />}
	\item \lstinline{<tpl for="" />}
	\item \lstinline{<tpl foreach="" />}
	\item \lstinline{<tpl if="" />}
	\item \lstinline{<tpl elseif="" />}
	\item \lstinline{<tpl else="" />}
\end{enumerate}


\subsection {general setup}

\paragraph {} Zoals kort beschreven worden template strings nu gegroepeerd en krijgt elke
template een naam. De systax is nu zo dat de templates wel geldige xml zijn (geen
rekening houdend met namespaces). Je kan de templates dan makkelijk in een xml file
gieten zonder dat tools en editors klagen over foute syntax.

Een store kan je dan maken door de urls op te geven naar de template bestanden. Bij het
maken van een template geef je dan de store mee waar de template zich in bevindt, en de
naam van de template die je wilt gebruiken.

\begin{lstlisting}[language=ownjavascript]
var store = Ext.create ('XmlTemplateStore', { templateFiles: ['templates.xml'] });
var template = Ext.create ('XCTemplate', {
	templateStore: store,
	templateName: 'include-test-main'
});
template.compile ();
\end{lstlisting}

De compile stap is optioneel, als je deze niet uitvoert wordt dit gedaan als je de
template de eerste keer toepast. Je kan de template nu gebruiken door ze toe te passen op
een object die de waardes bevat voor de template (\lstinline{template.apply (values)}).


\subsection {Werking van \lstinline {values}}

\paragraph {} Hierboven hebben we de grootste motivaties aangehaald om de template engine
van Sencha uit te breiden. We hebben er geen parameters of externe waarden gebruikt.
Natuurlijk is het noodzakelijk dat je wel variablen kan meegeven bij het uitvoeren van je
template. We zijn hier licht afgeweken van de manier waarop Sencha zijn expressies voor
variablen behandeld.

\paragraph {} Wanneer een template start kan je alle waarden terug vinden in het
\lstinline{values} object (net zoals sencha). Maar de \lstinline|{ ... }| syntax word niet
ondersteunt. De beperkingen van deze syntax worden eerder al aangehaald. Je verliest een
deel van de context (\lstinline{parent.parent} bestaat niet) je 2 for lussen in elkaar
nest (een veel voorkomend bijvoorbeeld is een tabel opbouwen, binnen elke rij voor elke
kolom word iets gedaan).

\paragraph {} In de nieuwe templates wordt \lstinline{values} niet veranderd wanneer we
in een for lus werken, maar kan je in de for lus aangeven hoe je de variable wilt
aanspreken (zie looping). \lstinline{values} blijft altijd verwijzen naar de waarde
waarmee de template wordt uitgevoerd.

\paragraph {} De syntax die wel nog ondersteunt wordt is \lstinline|{[ ... ]}| (resultaat
van de expressie opnemen in de uitput), en \lstinline|| (expressie uitvoeren maar
het resultaat niet opnemen in de output. Deze gedragen zich zoals voorheen. Je kan op deze
manier extra variabelen toegankelijk maken.

\begin{lstlisting}[language=xml]
<template name="main">
	
	
	...
	<tpl if="manyUsers">
		<a href="#users/2">next</a>
		<a href="#users/{[ nrOfPages ]}">last</a>
	</tpl>
</template>
\end{lstlisting}

\paragraph {} De manier om informatie op te vragen en te gebruiken is op deze manier
consistent. In de normale Sencha templates is dit minder gestroomlijnd \lstinline|{naam}|,
maar \lstinline|{[values.naam]}|, \lstinline|{parent}| en ook \lstinline|{[parent]}|,
\lstinline|{#}| maar \lstinline|{[xindex]}|, ...


\subsubsection {Include (\lstinline{<tpl include="..." values="..." />})}

\paragraph {} Dit statement zal in dezelfde store zoeken naar een template met de naam als
de waarde van het \lstinline{include} attribuut, en deze template uitvoeren op de plaats
waar dit statement voorkomt.

\paragraph {} Belangrijk is om de \lstinline{values} door te geven waarmee je wilt dat de
subtemplate zal werken. Doe je dit niet, dan zal de subtemplate helemaal geen waardes
hebben en dus ook geen juist resultaat tonen.

\paragraph {} Dat je deze waardes kan meegeven geeft je veel flexibiliteit. De main
template meot niet beginnen met waarden die in de subtemplate verwacht worden. Je kan ook
dezelfde sub template meerdere keren gebruiken maar met andere waarden.

\begin{lstlisting}[language=xml]
<template name="main">
	Hello <tpl include="sub" values="values.user" />!
</template>

<template name="sub">
	{[ values.firstname ]} {[ values.lastname ]}
</template>
\end{lstlisting}

\begin{lstlisting}
{
	"user": {
		"firstname: "John"
		"lastname": "Doe"
	}
}
\end{lstlisting}

\begin{lstlisting}
Hello John Doe!
\end{lstlisting}


\subsubsection {Extend and Block (\lstinline{<tpl extend="..." />}, \lstinline{<tpl block="..." values="..." />})}

\paragraph {} Het \lstinline{<tpl extend="..." />} statement geeft aan dat de structuur
van de template overeenkomt met die van de template vernoemt door de waarde van het
\lstinline{extend} attribuut. Dit betekent dat de template exact het zelfde is maar dat de
blocken in de hooft template overschreven worden.

\paragraph {} Een block in een template wordt aangegeven met een
\lstinline{<tpl block="..." />} statement. In een template die een andere extend is de
inhoud van zo een block een deel van de template die wordt uitgevoerd. Voor een template
die niet extend geeft een block aan waar deze template zal verschillen voor templates die
extenden.

\begin{lstlisting}[language=xml]
<template name="structure">
	Hello <tpl block="user" values="values.user" />!
</template>

<template name="colloquial">
	<tpl extend="structure" />
	<tpl block="user">
		{[ values.firstname ]}
	</tpl>
</template>

<template name="formal">
	<tpl extend="structure" />
	<tpl block="user">
		Mister {[ values.lastname ]}
	</tpl>
</template>
\end{lstlisting}

\paragraph {} Net zoals voor het \lstinline{<tpl include="..." />} statement moet je in de
main template opgeven welke waardes er worden doorgegeven aan de blocken die uiteindelijk
zullen worden uitgevoerd. Doe je dit niet, dan resulteert dit net zoals voor een include
in een foutief resultaat. De waardes moeten niet meer worden opgegeven voor de template
die de main template extend.

\paragraph {} Als je een andere template extend kan je geen nieuwe blocken toevoegen, je
kan alleen blocken die in de main template werden gedefinieerd overschrijven.

\paragraph {} Een block kan ook een default waarde krijgen voor het geval dat een andere
template deze block niet overschrijft. Dit is handig als een block vaak dezelfde waarde
heeft, maar in sommige gevallen toch aangepast moet worden.

\begin{lstlisting}[language=xml]
<template name="structure">
	Hello <tpl block="user" values="values.user">World</tpl>!
</template>

<template name="main">
	<tpl extend="structure" />
</template>
\end{lstlisting}

\begin{lstlisting}
Hello World!
\end{lstlisting}

\paragraph {} Het is mogelijk om een template die een andere template extend zelf ook te
extenden. Het komt waarschrijnlijk niet zo vaak voor dat je aan deze fuctionaliteit
behoefte hebt, maar de implementatie liet dit automatisch toe. Conceptueel worden de
invullingen van de blocken eerst gezocht in de template die wordt uitgevoerd, dan in de
parent van deze template, daarna in de parent van deze parent, enzovoort.

\paragraph {} In een block kan je ook andere templates includen. Deze templates kunnen
zelf templates zijn die een derde template extenden.

\begin{lstlisting}[language=xml]
<template name="main">
	<tpl extend="blog" />
	<tpl block="menu">...</tpl>
	<tpl block="footer">...</tpl>
</template>

<template name="blog">
	<tpl extend="structure" />
	<tpl block="body">
		{[ values.blogpost.title ]}
		{[ values.blogpost.body ]}
		<tpl include="blogcomments" values="values.blog.comments" />
	</tpl>
</template>

<template name="structure">
	<tpl block="menu" values="values" />
	<tpl block="boby" values="values" />
	<tpl block="footer" values="values" />
</template>

<template name="blogcomments">
	<tpl extend="comments" values="" />
	<tpl block="top">
		<form class="comment-form">
			...
		</form>
	</tpl>
</template>

<template name="comments">
	...
</template>
\end{lstlisting}

\paragraph {} Je kan in een block ook andere templates includen. Regelmatig wil je in een
template aangeven dat een bepaalde block eigenlijk gewoon een andere template is die je
include. Hiervoor is wat syntactische suiker voorzien.

\begin{lstlisting}[language=xml]
<template name="main">
	<tpl extend="structure" />
	<tpl block="body" include="body" />
</template>

<template name="structure">
	<tpl block="boby" values="values" />
</template>

<template name="body">
	This is the content.
</template>
\end{lstlisting}

\paragraph {} Het is bedoeling om in een template die een andere template extend
enkel de blocken te overschrijven, en geen andere template stukken toe te voegen. Het is
ook niet de bedoeling om eet template twee templates te laten extenden. Ook zou het
\lstinline{<tpl extend="..." />} statement voor eender welk \lstinline{<tpl block="..." />}
moeten komen. De volgorde van de blocken onderling maakt niet uit.


\subsubsection {Looping (\lstinline{<tpl for="..." as="..." />}, \lstinline{<tpl foreach="..." key="..." value="..." />})}

\paragraph {} Net zoals de standaard templates laat de \lstinline{<tpl for="..." />} je
toe om over een array te itereren. Maar in tegenstelling tot de standaard template wordt de
waarde van \lstinline{values} niet aangepast. In plaats daarvan geef je aan onder welke
naam je de waarden wilt gebruiken. Dat doe je door in het \lstinline{as} attribuut een
string in te vullen (dit moet een geldige javascript identifier zijn).

\paragraph {} De index van de huidige iteratie vind je dan onder de zelfde naam gevolgd
door \lstinline{Index}. Net zoals de standaard templates begint deze index vanaf 1 te
tellen. Geef je geen naam op bij het attribuut \lstinline{as} dan zal je template ook niet
werken.

\begin{lstlisting}[language=xml]
<template name="main">
	<tpl for="values.names" as="name">
		{[ nameIndex ]}) Hello {[ name ]}!
	</tpl>
</template>
\end{lstlisting}

\begin{lstlisting}
{
	"names": [ "john", "jane" ]
}
\end{lstlisting}

\begin{lstlisting}
1) Hello john!
2) Hello jane!
\end{lstlisting}

\paragraph {} Op een gelijkaardige manier kon je over objecten itereren met
\lstinline{<tpl foreach="..." />}. Hier geef je een naam voor de key
die je wilt gebruiken en een naamvoor de value.

\begin{lstlisting}
<template name="main">
	<tpl foreach="values.configuration" key="configKey" value="configValue">
		{[ configKey ]} is {[ configValue ]}
	</tpl>
</template>
\end{lstlisting}

\begin{lstlisting}
{
	"configuration": {
		"indent": "spaces",
		"level": 4
	}
}
\end{lstlisting}

\begin{lstlisting}
indent is spaces
level is 4
\end{lstlisting}

\paragraph {} Zo kan je de loops zo diep nesten (al is dit vaak geen goed idee) zonder dat
je ooit een deel van je context verliest.

\begin{lstlisting}
<template name="main">
	<tpl for="values.users" as="user">
		<tpl for="user.friends" as "friend">
			<tpl for="friend.posts" as "post">
				you gave {[ post.title ]} posted by {[ friend.name ]}
				a {[ post.ratings[user.id] ]} out of 5
			</tpl>
		</tpl>
	</tpl>
</template>
\end{lstlisting}


\subsubsection {Control flow (\lstinline{<tpl if="..." />},
	\lstinline{<tpl elseif="..." />}, \lstinline{<tpl else="..." />})}

\paragraph {} De statements voor de control flow werken zoals je zou verwachten. Het
verschil met de standaard Sencha templates is dat nu de expressies voor de attrituben
exact dezelfde zijn als de expressies die je zou gebruiken in \lstinline|{[ ... ]}|.

\paragraph {}Als je \lstinline{<tpl elseif="" />} of \lstinline{<tpl else="" />} gebruikt
mag er geen text staan tussen het begin van de \lstinline{else} of \lstinline{elseif} en
de vorige \lstinline{if} of \lstinline{elseif} statement. Je mag hier enkel whitespace
tussen voegen.

\begin{lstlisting}
<template name="main">
	<tpl if="values.type === 'normal'">
		...
	</tpl>
	<tpl elseif="values.type === 'failing'">
		...
	</tpl>
	<tpl elseif="values.type === 'successful'">
		...
	</tpl>
	<tpl else="">
		...
	</tpl>
</template>
\end{lstlisting}

\paragraph {} Het is niet nodig om deze constructies te gebruiken. Je zou ook gewoon
gebruik kunnen maken van \lstinline||,
\lstinline|| en \lstinline||. Maar de
\lstinline{<tpl />} versies zien er eleganter uit.


\subsubsection {Bemerkingen}

\paragraph {} In principe is het mogelijk om in de template waarden toe te voegen aan het
\lstinline{values} object. Deze waarden zijn dan ook toegankelijk in andere code, en kan
dus een invloed hebben op de werking van andere code. In normale situaties gebruik je deze
mogelijkheid niet. Het maakt je code moeilijk (of onmogelijk) te volgen. Er wordt geen
poging ondernomen door de implementatie van de template om dit anti patroon te voorkomen.
Maar een beetje gezond verstand zou voldoende moeten zijn om dit niet te misbruiken.


\paragraph {} Het is ook mogelijk (als je naar de implementatie gaat kijken om de blocken
(\lstinline{_ownedBlocks}, en \lstinline{_blocks} variable)
van template aan te passen. Er zijn nog een aantal variable die door de implementatie
gebruikt worden die mogelijk kunnen aangepast worden door code in de template.
Bijvoorbeeld de \lstinline{_out} array die de output van de template bevat, of de
\lstinline{_i0} variable die de index van een for loop bevat. Ook hier is het niet de
bedoeling om deze waarden aan te passen. Ook hier wordt geen poging onder nomen om dit
misbruik te stoppen.

\paragraph {} Alles Dat in global scope (attributen van het \lstinline{window} object op
web paginas) berijkbaar is kan door de template gebruikt worden, maar ook aangepast. Voor
het gebruik kan dit van pas komen. Zo heb je bijvoorbeeld toegang tot
\lstinline{Ext.util.Format} om datums en getallen te formateren. Maar Het is niet de
bedoeling om waarden in global scope aan te maken of aan te passen. Ook dit anti
patroon word niet tegengehouden.

\paragraph {} Dit is geen beperking van de nieuwe implementatie. Ook de templates die je
op de normale manier in Sencha maakt kan je misbruik maken van global scope, enz.

\paragraph {} Niet alle whitespace in de template wordt op dezelfde manier behandeld. Een
stuk text dat enkel bestaat uit whitespace en een cariage return bevat wordt volledig
genegeerd.

\begin{lstlisting}[language=xml]
<tpl for="values.users" as="user">
	{[ user.firstname ]}
	{[ user.lastname ]}
</tpl
\end{lstlisting}

geeft als resultaat

\begin{lstlisting}
JohnDoe
\end{lstlisting}

in plaats van 

\begin{lstlisting}

	John
	Doe

\end{lstlisting}

\paragraph {} Meestal is dit geen probleem. De templating engine is bedoelt om html
tegenereren. Voor html heeft de hoeveelheid Whitespace geen belang, zolang er ten minste
een spacie (of ander whitespace character) staat. Als je text op een nieuwe regel wil
laten beginnen moet je \lstinline{<br />} toevoegen, wat geen whitespace meer is.

\paragraph {} De enige plaats waar er problemen kunnen opduiken is in een situatie zoals
hierboven. Je plaatst twee expresies op hun eigen lijn, en hoewel de het resultaat wel in
de output verschijnt is alle whitespace er tussen verdwenen. Je zou door de template te
lezen wel verwachten dat de voornaam en achternaam van elkaar gescheiden worden.

\paragraph {} In deze situaties kan je whitspace tussen de expressies plaatsen door
\lstinline|| voor de laatste regel te plaatsen, of achter de eerste. De twee
expressies op die lijn worden dan enkel gescheiden door witruimte die geen enter bevat, en
wordt wel opgenomen in de output.

\begin{lstlisting}[language=xml]
<tpl for="values.users" as="user">
	{[ user.firstname ]} 
	{[ user.lastname ]}
</tpl
\end{lstlisting}

geeft

\begin{lstlisting}
John Doe
\end{lstlisting}
