\section* {Abstract}

\paragraph{}EVA is een langdurig project dat door Crossroad Communications NV wordt uitgewerkt op vraag van pcfruit. EVA staat voor Eindelijk Vereenvoudigde Administratie en zal fruittelers helpen met het vereenvoudigen met de administratieve taken. Eén van de onderdelen van EVA is de 'registratie en toepassing controle'. Deze module zal zich toespitsen op het ingeven en valideren van bemestings opdrachten en het toepassen van verschillende producten ter bestrijding van plagen en ziekten.
\paragraph{}Bij het uitvoeren van een bemesting of sproeiopdracht moet de teler rekening houden met de wettelijk toegelaten hoeveelheden en doseringen van producten. De wetgeving hieromtrent is zeer uitgebreid en complex waardoor de telers het overzicht kunnen verliezen. Zo is het bijvoorbeeld niet voldoende de juiste dosering te kiezen bij een sproeiopdracht maar moet de teler ook rekening houden met hoevaak er al werd gesproeid. Verder spelen er nog factoren mee zoals het totaal gebruikte volume en de mogelijkheid dat bepaalde producten niet samen gebruikt kunnen worden. Het wordt nog ingewikkelder wanneer het bloeistadium van de struik of boom limitaties oplegt. Voor alle uitgevoerde sproeibeurten moet de teler de correcte administratieve informatie beschikbaar maken om voor te leggen aan controleurs en veilingen. Het feit dat niet alle veilingen deze informatie in hetzelfde formaat wensen aangeleverd te krijgen maakt het er niet makkelijker op voor de teler.
\paragraph{}De ontwikkelde module zal de telers hepen door de administratieve taken te vereenvoudigen. Zo worden de telers geholpen door de sproei en bemestingsopdrachten in te geven en te valideren om zo foute toepassingen te vermeiden. De module helpt tevens ook om de informatie van alle opdrachten te bundelen en samen te vatten in de rapporten die de controlleurs en de verschillende veilingen nodig hebben.
\paragraph{}Dit eindwerk zal zich focussen op het verloop van de implementatie van deze module. Het project volgt de scrum methodology en wordt uitgewerkt in sprints van twee weken over een totale tijdspanne van 9 weken. Na deze 9 weken zal pcfruit een publieke demonstratie van deze module geven op fructura\footnote{www.fructura.be}

Tijdens de uitwerking van de module wordt gebruik gemaakt van een aantal frameworks en
tools. In dit eindwerk worden zowel de positieve als negatieve ervaring met deze tools
beschreven. Om deze bevindingen te illustreren worden ook representatieve en gedetailleerde
voorbeelden gegeven.
