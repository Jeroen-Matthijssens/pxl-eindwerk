\section* {Abstract}

\paragraph {} EVA is een langdurig project dat door Crossroad Communications NV wordt
uitgewerkt op vraag van pcfruit. EVA staat voor Eindelijk Vereenvoudigde Administratie en
zal fruittelers helpen met het vereenvoudigen van de administratieve taken. Eén van de
onderdelen van EVA is de 'registratie en toepassing controle'. Deze module zal zich
toespitsen op het ingeven en valideren van bemestingsopdrachten en het toepassen van
verschillende producten ter bestrijding van plagen en ziekten.

\paragraph {} Bij het uitvoeren van een bemesting of sproeiopdracht moet de fruitteler rekening
houden met de wettelijk toegelaten hoeveelheden en doseringen van producten. De wettelijke
normen zeer uitgebreid en complex waardoor de fruittelers het overzicht kunnen
verliezen. Zo is het bijvoorbeeld niet voldoende de juiste dosering te kiezen bij een
sproeiopdracht maar moet de fruitteler ook rekening houden met hoe vaak er al werd gesproeid.
Het wordt nog
ingewikkelder wanneer het bloei stadium van de struik of boom beperkingen oplegt. Voor alle
uitgevoerde sproeiopdrachten moet de fruitteler steeds de correcte administratieve informatie
beschikbaar maken om voor te leggen aan controleurs en veilingen. Niet alle veilingen
aanvaarde deze informatie in hetzelfde format, wat meer administratie voor de telers
oplevert.

\paragraph {} De ontwikkelde module zal de fruittelers helpen door de administratieve taken te
vereenvoudigen. Zo worden de fruittelers geholpen bij het ingeven van sproei- en
bemestingsopdrachten door deze te valideren om zo foute toepassingen te vermijden. De
module helpt om de informatie van alle opdrachten te bundelen en samen te vatten in de
rapporten die de controleurs en de verschillende veilingen nodig hebben.

\paragraph {} Dit eindwerk zal zich focussen op het verloop van de implementatie van deze
module. Het project volgt de scrum methodologie en wordt uitgewerkt in sprints van twee
weken over een totale tijdspanne van 9 weken. Na deze 9 weken zal pcfruit een publieke
demonstratie van deze module geven op fructura\footnote{www.fructura.be}.

\paragraph {} Tijdens de uitwerking van de module wordt gebruik gemaakt van een aantal
frameworks en tools. In dit eindwerk wordt in detail de positieve en negatieve aspecten
van deze tools besproken. Eventuele alternatieven voor de verschillende tools worden ook
onder de loep genomen. Om onze bevindingen te illustreren worden steeds representatieve en
gedetailleerde voorbeelden gegeven.
