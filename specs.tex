\section {Module fruitteelt beheer en toepassing controle}

\subsection {Web App}

\paragraph {} De web app is bedoeld om de fruittelers thuis de opdrachten op te stellen,
de uitgevoerde opdrachten te valideren, en de rapporten te genereren die hij moet kunnen
voorleggen.

Voordat een fruitteler de applicatie echt kan gebruiken is er eerst informatie nodig die
hij moet ingeven, dit is slechts 1 keer nodig bij het opstarten van zijn account. Een
overzicht van wat dit inhoud wordt gegeven in \ref{general_management} 'General
Managament'.

Eenmaal de startup is afgerond kan de applicatie gebruikt worden om (op dit moment) de
spuitopdrachten en bemestingsopdrachten op te stellen. Hierbij wordt de fruitteler
bijgestaan door de applicatie, die nu intelligente beslissingen kan maken en relevante
suggesties kan geven aan de teler. Dit wordt besproken in \ref{aanmaken_opdrachten}
'Aanmaken van opdrachten' en \ref{rapportering} 'Rapportering'.

pcfruit kan gebruik maken gewasbeschermingsmiddelen beheer. Zo kan pcfruit de wettelijke
beperkingen up to date houden.


\subsubsection {General Managament}\label{general_management}

\paragraph {} Je kan niet zelf geen account registreren via de web app (of de mobile app).
pcfruit maakt voor een nieuwe
fruitteler een bedrijf aan met een gebruiker. De fruitteler kan dan basis gegevens voor
zijn bedrijf aanvullen en personeelsleden aanmaken. De nieuwe personeelsleden kunnen
optioneel ook een gebruikers account krijgen op het platform. Die gebruikers zijn
automatisch gekoppeld aan het bedrijf van de fruitteler. Dit is bedoelt om werknemers
toegang te geven tot de mobile app zodat ze de opdrachten die de fruitteler aanmaakt als
afgehandeld kunnen markeren terwijl de werknemer zich nog op het veld begeeft.

\paragraph {} Om functioneel met de applicatie te kunnen werken moet de teler bepaalde
informatie opgeven. Dit maakt deel uit van de opstart van een bedrijf.
De applicatie moet weten welke percelen de teler bezit.
Elk perceel heeft een teelt die geteeld word en een oppervlakte. Als er houtig
kleinfruit of druiven worden geteeld op het perceel zal de teler ook moeten opgeven wat de
totale lengte is van alle rijen die op het perceel staan, samen met een standaard hoogte.
Dit wordt later gebruikt om de oppervlakte loofwand te berekenen. Telers van pit- en
steenfruit werken op een andere manier. Voor hun wordt dit berekened op basis von de
grond oppervlakte.

Percelen kunnen onderverdeeld worden in groepen. Vaak worden een groot aantal percelen in
een keer behandeld. Door alles in betekenis volle groepen te verdelen kan de fruitteler
makkelijker in een keer de relevante percelen selecteren.

Verder kan de fruitteler nog extra informatie over de percelen in geven die later van pas
kan komen. Er is ook de mogelijkheid om het perceel op een kaart te tekenen (google maps).
In principe zou je de oppervlakte van het perceel kunnen berekenen uit het getekende
figuur. Maar in praktijk weet de teler exact hoe groot zijn perceel is.

Het is wel nuttig om de geografische locatie te kennen om later een voorstel te doen op
basis van gps co\"ordinaten. Als een gebruiker een registratie wil invoeren via de mobile
app, dan kunnen we de lijst van percelen bijvoorbeeld rangschikken op basis van afstand
tot de plaats waar de gebruiker zich op dat moment bevindt.

\paragraph {} Bij het uitvoeren van een spuitopdracht wordt altijd een machine gebruikt.
Deze heeft een tankvolume en verbruikt een hoeveelheid water per oppervlakte. De
toestellen bij verschillende opdrachten opnieuw gebruikt. Hier kan de fruitteler de
informatie een keer ingeven en moet hij zich er niets meer van aantrekken bij het aanmaken
van een opdracht.

Er zijn een aantal verschillende type van toestellen. Sommige toestellen worden gebruikt
om de loofwand van de planten te besproeien, sommige machines worden gebruikt om onkruiden
op de grond te behandelen of om meststoffen te verspieden. Deze toestellen hebben andere
informatie waarin de applicatie ge\"interesseerd is. Als je het type van het toestel
veranderd, worden er andere velden zichtbaar die ingevuld kunnen worden.

\paragraph {} Als een fruitteler een bemestingsopdracht wilt opstellen moet hij ook
producten selecteren, maar de lijst van mogelijke producten zijn eindeloos. pcfruit heeft
wel een lijst samengesteld met veelgebruikte meststoffen, maar er mogen ook andere
producten worden gebruikt. De fruitteler kan bij de meststoffen zijn eigen producten
toevoegen. Producten die een teler zelf toevoegt zijn niet zichtbaar voor andere telers.


\subsubsection {Aanmaken van opdrachten}\label{aanmaken_opdrachten}

\paragraph {} Je krijgt een lijst van alle opdrachten die al aangemaakt zijn. Deze kan je
dan selecteren en aanpassen, of je kan een volledig nieuwe opdracht aanmaken. Bij de
bestaande opdrachten krijg je standaard het overzicht te zien. Bij het aanmaken van een
opdracht moet de volgorde van de tabs bovenaan gerespecteerd worden. Doe je dit niet, dan
kan je geen waarden ingeven. De informatie in deze tabs hangen af van wat in de vorige
werd ingegeven. Als je informatie in de voorgaande aanpast gecontroleerd of alles nog
klopt. Is dit niet het geval dan wordt de gebruiken gewaarschuwd en kan hij kiezen om het
systeem alles dat niet klopt te verwijderen, of om de verandering ongedaan te maken.

\paragraph {} Als je een opdracht aanmaakt kan je een naam opgeven. De naam dient alleen
om makkelijk je opdracht terug te vinden in de lijst. Je geeft aan wat voor type opdracht
je wilt maken, bemesting, gewasbescherming, onkruidbestrijding of andere. Deze keuze is
belangrijk om te bepalen welke toestellen gebruikt kunnen worden, en welke producten de
fruitteler wil selecteren.

\paragraph {} Je geeft ook op voor welke teelt groep de opdracht van toepassing is.
Fruittelers uit de verschillende groepen zijn gewoon om met andere eenheden te rekenen.
Het is belangrijk om de telers waarden en berekeningen te tonen die overeenkomen met hun
manier van werken.

Zo wordt in houtig kleinfruit de werkelijke loofwand oppervlakte berekend en gebruikt in
verdere berekeningen. Telers var pit- en steenfruit gebruiken een omzet factor om de
loofwand oppervlakte te benaderen.

Appelbomen en perenbomen blijven jaren op een perceel staan en veranderen niet van hoogte,
de breedte van de rijen en de afstand tussen de rijen blijft ongeveer dezelfde van perceel
to perceel. Je kan dus een keer de verhouding tussen de loofwand oppervlakte en de
grondoppervlakte kunnen uitrekenen en daarna altijd deze factor gebruiken. Er is immers
weinig verschil tussen de percelen onderling en verschillende tijdstippen op een jaar voor
een percelen.

Houtig kleinfruit zijn vaak planten die snel groeien en in potten geteeld worden.  Er is
dus een groot verschil tussen de planten wanneer ze net op het veld worden geplaatst en de
planten net voor de oogst. Als je in het begin bijvoorbeeld $1.5$ keer de grondoppervlakte
gebruikt wordt om de dosis te bepalen, dan zal je de planten met teveel product
behandelen, en later net voor de oogst te weinig. Voor de telers is de oppervlakte
loofwand wel degelijk belangrijk.

Druivelaars zijn ook bomen die lange tijd op een perceel blijven staan en ongeveer
dezelfde hoogte behouden. In tegenstelling tot pit- en steenfruit wordt hier niet met een
factor gewerkt, maar wordt de oppervlakte loofwand ook berekend.

\paragraph {} Als je alle voorgaande informatie hebt ingegeven worden de machines die bij
de opstart werden ingegeven gefilterd zodat enkel relevante machines getoond worden. De
fruitteler selecteert er een van. Hij kan daarna nog altijd de hoeveelheid water dat
gebruikt wordt aanpassen. Het tank volume van een machine blijft wel dezelfde maar de druk
en de sproei knoppen kunnen aangepast worden om meer of minder water te verbruiken.

\paragraph {} Na het selecteren van de machine dat gebruikt wordt voor de opdracht kiest
de fruitteler welke percelen behandeld moeten worden. Deze lijst is al gefilterd op basis
van de teelt groep die eerder werd gekozen.

\paragraph {} Als laatste worden de producten voor de tankmix gekozen. Welke producten de
teler kan kiezen hangt af van het type opdracht dat wordt gemaakt. Voor onkruidbestrijding
en gewasbeschreming moet de teler ook opgeven waarom hij het product gebruikt. Welke plaag
of ziekte hij wil bestrijden. Voor deze producten worden de dosissen voorgesteld. Dit zijn
producten waarop de wettelijke limieten van toepassing zijn.

Voor gewasbescherming (dit zijn opdrachten waar de loofwand van de teelt behandeld wordt)
kan de fruitteler ook bepaalde meststoffen in de tankmix toevoegen. Hier zijn geen
voorstellen voor dosering mogelijk en de fruitteler moet ze dus zelf ingeven.

\paragraph {} Nadat alle details zijn ingeven kan je op het overzicht zien hoeveel van welk
product nodig is om alle percelen te behandelen. Er wordt een berekening gemaakt zodat de
fruitteler weet hoeveel volle tanken hij moet mixen, en hoe vol zijn laatste rest tank
moet zijn. Voor zowel de volle tanken als de rest tank wordt wat de totale dosis getoond
die in de tank moet worden toegevoegd. Dit zijn berekeningen die hiervoor met pitsoft
gebeurde, of met een rekenmachine en een stuk papier.

\paragraph {} Na het overlopen van het overzicht kan de fruitteler de nieuwe opdracht
opslaan. Voor dat dit echt wordt bijgehouden krijgt de teler een rapport te zien met
eventuele waarschuwingen als producten bijvoorbeeld te vaak gebruikt wordt. Producten
mogen maar een beperkt aantal keer gebruikt worden op hetzelfde perceel. Niet alleen mogen
de producten een beperkt aantal keer gebruikt worden, maar de producten hebben actieve
stoffen die zelf ook maar een beperkt aantal keer mogen gebruikt worden. De actieve
stoffen worden onderverdeeld in families. Weer mogen bepaalde families slechts een aantal
keer worden ingezet.

Je zou dus een product vier keer mogen gebruiken, maar omdat je een ander product hebt
gebruikt met actieve stoffen in de zelfde families zou je nu bij de tweede toepassing al
een overtreding begaan. Deze informatie wordt in een overzicht weergegeven met
aanduidingen als beperkingen zouden worden overschreden.

De teler kan dan kiezen om terug naar de opdracht te gaan en zijn producten of percelen
aan passen zodat hij niet over de wettelijke beperkingen gaat, of hij kan kiezen om toch
op te slaan en later de opdracht nog aanpassen.

\paragraph {} De opdracht die zijn opgeslagen kan dan via de mobile app of via de webapp
worden uitgevoerd.


\subsubsection {Schemas}

\paragraph {} Nadat de opdrachten werden uitgevoerd zijn ze zichtbaar in een schema. Hier
kan de fruitteler nog eens over alle opdrachten en gaan en controleren of alles
geregistreerd is zoals op het veld gebeurde. Hij kan dan de opdrachten valideren, en het
zijn de gevalideerde opdrachten die worden opgenomen in de rapportering.

\paragraph {gewasbescherming}

\paragraph {bemestingen} De wetgeving wordt herzien tijdens de ontwikkeling van de
applicatie. Het is dus nog niet mogelijk om voor overtredingen te waarschuwen. In het
overzicht wordt wel getoond hoeveel stikstof, fosfor, en kalium er bij elk product word
gebruikt.

\paragraph {toepassingen}


\subsubsection {rapportering}\label{rapportering}

\paragraph {} Gelijkaardig aan het schema kan je hier een overzicht vinden van alle
opdrachten die gevalideerd werden.

\paragraph {} Er is ook een overzicht dat de telers helpt om te zien hoevaak een bepaald
product op een perceel gebruikt werd, en of er daarbij een limit werd overschreden.


\subsubsection {gbmbeheer}

\paragraph {} Om de informatie rond het gebruik van producten en hun wettelijke
beperkingen up to date te houden heeft pcfruit een een account waarop deze informatie kan
worden aanpepast.


\subsection {Mobile app}

\subsubsection {timeline}

\subsubsection {mobile app registreren}

\paragraph {} De belangrijkste opdrachten worden in de tijdslijn opgelijst. De gebruiker
kan op de opdracht klikken en krijgt een schrem waar hij details kan aanpassen. De
opdrachten die in de tijdslijn worden opgelijst zijn de spuitopdrachten en de bemestings
opdrachten. Er zijn twee andere registraties die kunnen worden ingegeven, oogst
registraties en taak registraties. Deze hebben voorlopig enkel een informatieve rol.

\paragraph {} Alle registraties volgen een gelijkaardig structuur. Alle registraties
hebben een lijst van percelen waarop de registratie wordt gedaan, en een lijst van
personeelsleden dat deelnam aan de activiteiten. Soms zijn niet alle personen die
deelnamen aan de activiteiten personeels leden. In dat geval is er de mogelijkheid om ook
op te geven hoeveel personen hebben deelgenomen, zonder dat er een link gemaakt wordt naar
de personeels leden.

Er is altijd een datum waarop de registratie wordt uitgevoerd, een duur die aangeeft
hoe lang de activiteiten in beslag namen, en een optionele opmerking waar de gebruiker
vrij wat text kan ingeven.

Afhankelijk van het type registratie zijn er soms enkele extra velden. Bij oogst kan het
type verpakking worden ingegeven (en het aantal natuurlijk).


\subsubsection {mobile app waarnemingen}

\paragraph {} Waarneming zijn gelijkaardig aan de registraties, maar ze hebben geen
gelinkte personeelsleden. De waarneming wordt gewoon gedaan door de persoon die ze
ingeeft. Er is wel een lijst met gelinkte percelen.

Waarnemingen zijn bijvoorbeeld valvangsten van een plaag. Een andere is de fenologisch
stadiums van hun teelt aangeven. Dit wordt in een latere fase gebruikt om de fruitteler te
waarschuwen als een product niet mag toegepast worden in het stadium waarin zijn teelt
zich op dit moment bevindt.


\subsection {gebruik local storage, rest backend}

\paragraph {} Een hoop data is vrij statisch vanuit het opzicht van de fruitteler. De
erkenningen en producten veranderen niet vaak. De data is wel redelijk groot en wordt op
veel plaatsen gebruikt.
