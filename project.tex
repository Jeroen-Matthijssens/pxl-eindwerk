\section {project}

\subsection {Kadering}

\paragraph {} Fruittelers moeten in het uitoefenen van hun beroep een hoop administratie
te verwerken. Zo worden hut fruitbomen en fruitstruiken regelmatig behandeld om de planten
te beschermen tegen plagen en ziekten. Deze behandelingen moeten geregistreerd worden.
Later moet de teler deze lijst kunnen voorleggen aan controleurs en veilingen. De
controleurs zullen nagaan of de fruitteler de wettelijke beperkingen hebben gerespecteerd.
De veilingen willen voornamelijk weten of er niet te veel residu op het fruit is achter
gebleven.

\paragraph {} Nu bestaat er een soort logboek waar de telers hun informatie moeten
bijhouden. De administratie gebeurt in op papier geprinte bestand. In praktijk wordt deze
registratie niet altijd even goed uitgevoerd. De adviseurs van pcfruit worden zo soms door
telers gecontacteerd die op het einde van het jaar niet het nodige papier werk in orde
hebben en de hulp van het advies centrum willen inschakelen.

\paragraph {} Voor de fruitteler is het niet altijd eenvoudig om alle wettelijke
beperkingen te kennen (hiervoor wordt de kennis en de expertise van het advies centrum
gebruikt). De fruittelers moeten wel hun fruitbomen behandelen, maar contacteren daarvoor
niet altijd pcfruit. Zo ontstaat de kans op fouten.

Voor het uitvoeren van een behandeling moet de teler eers altijd een berekening maken om
te weten hoeveel van welk product hij in een tank moet doen, en hoeveel volle tanken hij
zo nodig heeft. Om overschot (en dus ook kosten) te reduceren wil de teler ook graag weten
hoe vol de laatste tank moet zijn, en hoeveel product er in deze laatste tank moet.

Voor dit project is er reeds door pcfruit een poging ondernomen om het registreren en het
berekenen te vereenvoudigen. Immers als een fruitteler toch een berekening moet maken, kan
je de applicatie deze berekening laten uitvoeren. De applicatie heeft dan alle gegevens
die nodig zijn om uiteindelijk in de rapportage op te nemen. Het programma dat gebruikt
wordt is PitSoft.

\paragraph {} PitSoft heeft een grote beperking. Het weet immers niet welke dosisen nodig
zijn om ziekten en plagen te bestrijden, en wat wettelijk toegelaten is. De fruitteler
moest dus eerst de juiste dosis opzoeken, dan ingeven in PitSoft, om uiteindelijk het
resultaat van de berekening te krijgen. Voor een aantal producten kennen de telers deze
dosissen natuurlijk uit het hoofd, maar zeker niet voor allemaal. Daarbij komt dat de
wettelijk toegelaten normen nog al eens kunnen veranderen. Voor het moeilijkste gedeelte
moest de teler nog altijd handmatig achter informatie zoeken.

Een extra beperking van PitSoft is dat het de fruitteler enkel ondersteunt voor
registraties voor behandelingen van hun perceel en bemestingen. Alle andere informatie die
verplicht in het logboek moet worden opgenomen wordt overgelaten aan de fruitteler. Zo
moet de fruitteler ook bijhouden welke ziekten en plagen hij heeft ontdekt op zijn
percelen, welke insecten vallen hij heeft opgezet, of in fonologisch stadium de
fruitbomen en fruitstruiken zich bevinden.

\paragraph {} EVA (Eindelijk vereenvoudigde administratie), het platform aangeboden door
pcfruit, is een nieuw project dat opgestart werd ter vervanging van PitSoft. EVA op een
moderne manier de tekortkomingen van PitSoft aan.


\subsection {Scope van het project}

\paragraph {} EVA is een project dat over een tijdspanne van vijf jaar wordt uitgewerkt.
In een eerste fase wordt voornamelijk de concrete vereisten van het systeem in kaart
gebracht door het maken van prototypes.

\paragraph {} Het project bestaat uit twee delen, een deel van de applicatie is
toegankelijk via een website, een ander gedeelte word aangeboden als een mobile
applicatie. De flow zou er ruwweg als volgt uitzien. De fruitteler maakt eerst een
opdracht aan, hij kiest hierbij het type (doel) van de opdracht, het toestel dat gebruikt
wordt om de opdracht uit te voeren, de percelen waarop de opdracht moet worden
uitgevoerd, en de producten waaruit de behandeling bestaat. De applicatie geeft dan een
overzicht met de berekende tankvolumes en product doseringen.

De fruitteler of een werknemer Kan dit overzicht afdrukken en de tankmix samenstellen,
wanneer hij op het veld klaar is met het uitvoeren van de opdracht kan hij een smartphone
of tablet nemen en de opdracht uitvoeren via de mobile applicatie. Later op de dag kan de
fruitteler dan verifi\"eren dat alles correct is verlopen en de opdracht definitief
registreren. De fruitteler kan daarbij nog wijzigingen aanbrengen als ze bijvoorbeeld
constateren dat niet de correcte dosis werd gebruikt bij het samenstellen van de tankmix,
of als er percelen niet werden behandeld zoals geplant.

Later als een controleur deze documenten wil inzien kan de fruitteler een rapport printen
dat alle behandelingen toont.


\subsection {Stakeholders}

\paragraph {crosscomm communications NV} De implementatie en de ondersteuning van deze
software zal gebeuren door Crosscom communications NV. Wij zullen de nodige technische
analyse uitvoeren. We zitten samen met pcfruit aan de tafel om de planning op te stellen
en de einddoelen per fase vast te leggen. Wij zullen de nodige hardware voorzien om het
platform op te draaien, en voorzien de nodige support om alles in goede banen te leiden na
de deployment van de fases. Wij zijn geen eindgebruiker van het platform, maar we de
technische kant van he project uitwerken.

\paragraph {pcfruit} De software wordt ontwikkeld voor pcfruit, dat heel het project
financiert. Zij zal als adviescentrum en productonwikkelaar op de hoogte zijn van elke
stap in de ontwikkeling van de software. Zij hebben de mogelijkheid om bij te sturen
de noden veranderen of duidelijker worden.

pcfruit gebruikt de applicatie om de correcte beperkingen op het gebruik van
producten te beheren. Op deze manier is de informatie die een teler krijgt altijd up to
date. De adviseurs van pcfruit kunnen in een later stadium de applicatie ook gebruiken om
telers beter te kunnen helpen. Zo zou een adviseur (mits toestemming van de teler) de
spuitopdrachten kunnen inkijken en alternatieven kunnen voorstellen.

pcfruit zal de feedback van telers en testers verzamelen en doorgeven aan Crossroad
Communicatiens NV.

\paragraph {Telers} De eindgebruikers van het systeem. De telers telen een of meerdere
typen fruit. Zij vormen de grootste groep eindgebruikers van het systeem. De noden van de
fruittelers zullen vertegenwoordigd worden door pcfruit. Rechtstreeks hebben de
fruittelers geen inspraak in de ontwikkeling van de applicatie. Om telers makkelijker te
overtuigen de nieuwe software te gebruiken voor het registreren on opstellen van
spuitopdrachten is het belangrijk genoeg aandacht te schenken aan de interface en
bruikbaarheid van de applicatie.

Dit is niet altijd eenvoudig. Er zijn een aantal verschillende groepen waaronder een teelt
kan vallen, houtig kleinfruit, pit- en steenfruit, aardbeien, druiven, ... In elke teelt
worden gelijkaardige dingen op een net iets andere manier gedaan. Voor aardbeien word
bladvoeding op dezelfde manier toegepast op het perceel als onkruidbestrijding, en wordt
in beide gevallen rekening gehouden met de grond oppervlakte voor het berekenen van de
tank volumes.

Voor houtig kleinfruit wordt de bladvoeding op de loofwand gespoten. Voor
onkruid wordt er echter op de grond herbicides gespoten. Hier moet voor de berekeningen
van onkruidbestrijdingen rekeninng worden gehouden met de grond oppervlakte, maar voor de
bladvoedingen met de oppervlakte loofwand.

In het geval van pit- en steenfruit worden de producten op dezelfde manier toegepast als
voor houtig kleinfruit, maar hier willen de telers altijd rekenenen met grond
oppervlaktes, niet met oppervlakte haag. Hoewel de wettelijke limiten enkel uitgedrukt
worden in oppervlakte haag.

Het is belangrijk om de teler op een vertrouwde manier informatie te tonen, maar tegelijk
ook om op een correcte uniforme manier berekeningen uit te werken.

Het is belangrijk om de feedback die telers geven over het gebruik van de applicatie te
verzamelen en te verwerken. Dit zal gebeuren door pcfruit. In de eerste plaats wordt de
applicatie gebruikt door werknemers van pcfruit in de proeftuin. In een tweede fase worden
stappen ook echte telers mee in het project. Deze telers hebben dan de mogelijkheid om mee
hun stepel de drukken op de nieuwe applicatie.

\paragraph {Overheid} Een groot deel van de applicatie draait rond rapportering, en in de
eerste plaats rapportering naar controleurs. Een ander gedeelte laat toe om telers op de
hoogte te brengen van de wettelijke beperking van het gebruik van pesticides en andere
producten. Rechtstreeks zal de overheid natuurlijk niet actief als partner mee spelen. Hun
noden wordt ook door pcfruit vertegenwoordigd. Zo moeten de rapporten die de applicatie
genereerd overzichtelijk genoeg zijn, moet ze de juiste informatie bevatten, en relevant
zijn voor de inspectie die controleurs uitvoeren.

Zo wordt er bijvoorbeeld nog altijd overlegd over De beperkingen die worden opgelegd voor
bemestingen. Eens er een beslissing wordt genomen zal pcfruit dit kaderen binnen het
project.


\subsection {Probleembeschrijving}

\subsubsection {Verkoop}

\paragraph {} Fruittelers verkopen hun fruit op een of meerdere fruitveilingen. Maar de
fruitveilingen willen weten welke producten gebruikt werden op het fruit. De teler moet
deze informatie doorgeven aan de veiling.

\paragraph {} Fruittelers willen op een zo simpel mogelijke manier hun fruit
verkopen op een veiling of op meerdere veilingen. Veilingen willen hun kwaliteit
garanderen en stellen hiervoor criteria op waar de producten - die geveild worden - aan
moeten voldoen. Deze vooropgestelde regels moeten overzichtelijk en up-to-date blijven
voor de fruittelers. De fruittelers moeten de criteria kunnen zien, en moeten kunnen
opvolgen of ze op elk moment voldoen aan deze criteria. Een fruitteler wil niet een heel
jaar fruit onderhouden, om het dan na de oogst niet te kunnen verkopen omdat ze een
criterium over het hoofd gezien hebben. Een goed voorbeeld hiervoor is de dosering van
pesticiden. Veilingen willen kunnen garanderen dat er ten hoogste een bepaalde hoeveelheid
pesticide gebruikt is op hun producten. De fruitteler moet dan eerst en vooral weten
dat deze regel bestaat. Hij zal ook moeten weten wat de limieten van de regel zijn.
Uiteindelijk zal hij dan ook een overzicht willen van de gebruikte pesticiden op zijn
perceel. Op die manier blijft hij een goede kijk hebben op heel het gebeuren, en zal hij
exact weten of hij op het einde van het jaar zijn oogst zal kunnen verkopen. Het is
vanzelfsprekend dat deze hoeveelheid aan informatie - en het feit dat de informatie toch
wel cruciaal is - een serieuze berg werk met zich meebrengt. Denk daarbij aan het feit dat
fruittelers niet noodzakelijk 1 perceel hebben. Verschillende percelen onderhouden
betekend ook verschillende percelen besproeien. Het zou zonde zijn als je 2 keer hetzelfde
perceel zou besproeien omdat je de informatie niet correct bijgehouden hebt. Fruittelers
verbouwen ook regelmatig verschillende soorten fruit. Niet alleen zal hij dan voor elk
soort fruit de regels moeten onthouden, maar hij zal ook voor elk soort fruit goed moeten
opletten en bijhouden hoeveel pesticide hij gebruikt. Heel deze 'boekhouding' kost enorm
veel werk en vraagt enorm veel moeite om dit allemaal manueel gaan bij te houden. Het zou
allemaal een heel stuk makkelijker zijn mocht er een centraal punt zijn waar alle
informatie gestructureerd wordt aangereikt en waar alle persoonlijke informatie rond het
produceren van fruit, per fruitteler, kan bijgehouden worden en ten alle tijde
geraadpleegd kan worden. Indien heel het platform dan toch centraal staat, wil dat ook
zeggen dat meerdere mensen gegevens kunnen uitwisselen indien ze dat willen. Op die manier
zal je ook in teamverband kunnen samenwerken en verschillende percelen tegelijk
onderhouden zonder dat er communicatieproblemen zijn.

\paragraph {} Een teler wil zijn oogst verkopen op een veiling waar strikte regulering is
omtrent de gewassen die verkocht worden. Zo is het bijvoorbeeld niet aanvaardbaar fruit te
verkopen waarop een te hoge dosis pesticide wordt aangetroffen. Het kost moeite om al deze
informatie bij te houden. Deze informatie moet dan voor elke veiling apart worden door
gegeven. Sommige veilingen werken met een elektronisch systeem waar de telers op kunnen
inloggen. Sommige veilingen werken met papieren aangiftes.

De informatie over welke producten werden gebruikt moet ook aan controleurs worden
voorgelegd. 
Als de informatie 

Deze informatie verzamelen gedurende het hele jaar kost tijd en moeite.
Het bijhouden van deze
informatie gedurende het hele jaar kost veel tijd en moeite voor een teler. Zonder hulp of
advies bestaat het risico dat een onvoldoende geïnformeerde teler zijn oogst niet verkocht
krijgt op de veiling, of zelfs boetes moet betalen.


\subsection {Oplossing}

\paragraph {} pcfruit zal een platform aanbieden waarop de fruittelers hun percelen en hun
gewassen zullen kunnen beheren. Fruittelers zullen hierop hun percelen virtueel kunnen
voorstellen en zullen op deze percelen allerhande virtuele acties kunnen ondernemen. Deze
acties zijn eigenlijk gewoon de virtuele counterparts voor de acties die de telers op hun
echte percelen toepassen. Gaat een fruitteler bemesten op een perceel, dan zal hij een
virtuele actie 'bemesten' aanmaken - met een aantal specificaties zoals: type bemesting,
type aanbrenging, ... etc. Deze acties zal hij dan kunnen terugvinden via eender welke
internet verbinding, waarna hij de acties als uitgevoerd kan aanduiden. Op deze manier kan
iemand taken aanmaken voor andere fruittelers. Op deze virtuele percelen zullen ook
virtuele gewassen aangebracht worden om op die manier de representatie van de
werkelijkheid zo realistisch mogelijk te houden. Na het inbrengen van de virtuele gewassen
zullen een aantal van de mogelijke acties - voor het virtuele perceel met dit virtueel
gewas - aangereikt worden om op die manier informatie te verschaffen aan de fruittelers.
[[[Op die manier heeft het bijvoorbeeld geen zin om pesticiden te selecteren om mestkevers
te bestrijden, als je een veld met bananenbomen hebt, wetende dat bananenbomen niet
gevoelig zijn voor mestkever]]]. Een hele lijst van pesticideacties voor het virtuele
gewas dat je op dit virtuele perceel hebt staan, zal daarentegen wel weergegeven worden om
selecties en acties makkelijk aan te maken. Bij het aanmaken van acties zal er ook altijd
rekening gehouden worden met alle acties die op het virtuele perceel gebeurd zijn. Op die
manier zal je een notificatie gegeven worden al je al voor de 3de keer je perceel gaat
bemesten, terwijl je bij de 2de bemesting eigenlijk al aan je totaal toegelaten
hoeveelheid meststoffen zat. Als de fruitteler daarentegen een actie toevoegd die hem de
gewassen van het perceel laat vervangen door een ander gewas, dan zal bijvoorbeeld zijn
teller voor pesticiden op het huidige gewas terug op 0 gezet worden. Na al deze informatie
bijhouden, weergeven en advies over de ingegeven informatie geven, kan je ook steeds
afdrukken raadplegen over de huidige situatie van een perceel of een gewas. Op die manier
zal de communicatie naar de verschillende veilingen en de overheid een stuk aangenamer
verlopen.


\subsection {Voordelen}

\paragraph {} Boven de oplossingen die de applicatie beid voor de problemen en ongemakken
die de fruittelers ervaren zijn er extra mogelijkheden om de software nog nuttiger te
maken.

\paragraph {} Omdat de informatie nu digitaal wordt bijgehouden is het mogelijk om later
nieuwe rapporten te genereren die anders gestructureerd zijn. De teler moet niets extra
doen om bijvoorbeeld zijn fruit te exporteren naar landen waar weer een andere manier van
rapportering wordt gebruikt.

\paragraph {} Het is nu ook mogelijk om data van een
veel telers samen te analyseren, bijvoorbeeld om bepaalde patronen van ziekte verspreiding
in kaart te brengen. Hierbij moet natuurlijk de privacy van de fruitteler gerespecteerd
worden (bijvoorbeeld door de data enkel te gebruiken wanneer de fruitteler toestemming
geeft). De crowdsourcing van de data die telers ingeven opent de deur voor een hoop
analyses die voorheen veel moeilijker of praktisch onmogelijk waren.

\paragraph {} Omdat het platform een online service is kan het gebruikt worden om telers
en adviseurs met elkaar te laten communiceren. De adviseurs kunnen de extra informatie die
uit de crowdsourcing naar boven komt gebruiken om fruittelers preventief informatie te
geven

\paragraph {} Stel dat er een aantal fruittelers die geografisch dicht bij elkaar percelen
beheren problemen krijgen met een kever plaag. Elk van deze fruittelers zal in het
platform registraties invoeren voor de plaag, en waarschijnlijk ook een behandeling
samenstellen om zijn teelt tegen de plaag te beschremen. Een van de andere fruittelers zal
mogelijk de plaag nog niet vast gesteld hebben. pcfruit kan door middel van de eerder
vergaarde data een suggestie sturen naar de fruitteler die deze plaag nog niet heeft vast
gesteld. De teler ontdekt dan sneller, in een vroeger stadium de plaag op zijn perelen, en
kan zo zijn fruitbomen beter en goedkoper beschermen.

\paragraph {} pcfruit  kan deze data gebruiken om de doeltreffendheid van behandelingen te
controleren wanneer ze in werkelijke situaties worden toegepast, of om te controleren met
welke bemesting strategie\"en het meeste oogst wordt gerealiseerd. Omdat pcfruit data van
meerdere fruittelers kan inkijken over verschillende jaren is het mogelijk om trends te
ontdekken die anders onzichtbaar bleven.

\paragraph {} 


\subsection {Something}


\paragraph {} Een eerste voordeel is dat de informatie van de sproeitaken digitaal wordt
bijgehouden. Hierdoor kan de fruitteler de overheid en de veiling voorzien van het
nodige papierwerk wanneer hun oogst voor verkoop wordt opgegeven. pcfruit voorziet de
teler van de nodige rapportages over alle sproei taken waarin de samenvatting van de
gebruikte stoffen te vinden is.

\paragraph {} Een tweede voordeel is dat pcfruit de fruitteler bijstaat bij het kiezen van de
stoffen die gebruikt kunnen worden tegen specifieke kwalen. Zo kan bijvoorbeeld gezocht
worden op welke producten helpen tegen zwart-rot. Zo kan de fruitteler steunen op de
uitgebreide databank van pcfruit om de gewassen te beschermen. Wanneer een fruitteler een
ziekte vast stelt bij de gewassen wordt hier ook notitie van genomen zodat pcfruit kan
helpen door middel van expert advies.

\paragraph {} Een derde voordeel wordt duidelijk wanneer de fruitteler een sproei opdracht
maakt. Tijdens het aanmaken kiest hij het perseel waarop gespoten gaat worden en het
toestel waarmee hij de sproei opdracht zal uitvoeren. Nadat de sproei opdracht is
aangemaakt kan het platform de dosering van de gebruikte producten berekenen. Zo kan de
teler zeker zijn dat hij altijd de juiste hoeveelheid stoffen gebruikt en bijgevolg ook
het juiste papierwerk kan voorleggen bij de volgende controle.


\subsection {Toekomstplannen}

Bemestingen

Gelijkaardig aan de spuit opdrachten zal pcfruit de juiste berekeningen en informatie
aanbieden voor bemestingen. Zo zullen fruittelers geen accidentele overbesmesting meer doen en
kunnen ze de juiste rapportering aanbieden aan de overheid. Het is bijvoorbeeld
belangrijk dat de verschillende stoffen de wettelijke limieten niet overschreiden.
Kalium, fosfor en stikstof zijn wettelijk gelimiteerd. Manueel de gebruikte hoeveelheden
nitraten beheren is lastig, en tijd consuming.

Communicatie met voorlichters en adviseurs

Next gen interfaces voor mobile applicatie

Wanneer een van de werknemers door de percelen aan het wandelen is en een probleem vast
stelt met de gewassen kan hij hier een notitie van nemen op het platform. Omdat we in een
mobile omgeving werken kunnen we gebruik maken van gps locatie om automatisch het perceel
te kiezen tijdens het invullen van de waargenomen problemen.

Crowdsourcing

\paragraph {} Een tweede voordeel van crowdsourced data te analyseren is dat de fruittelers ook
kunnen opgeven hoeveel oogst ze hebben gehaald uit verschillende gewassen. Omdat alle
sproeiopdrachten die de fruitteler gedaan heeft tijdens dit jaar beschikbaar zijn, kunnen er
trends ontdekt worden in de data. Zo kan er bijvoorbeeld gemerkt worden dat een van de
telers een zeer goede oogst hebben gekregen, terwijl andere fruittelers slechts een gemiddelde
oogst hadden. Na data analyse kan er dan geconstateerd worden dat deze fruitteler tegen een
bepaalde kwaal geproeid heeft die de andere fruittelers over het hoofd gezien hadden. Met deze
informatie kan er in het volgende jaar een suggestie gegeven worden aan de fruittelers wanneer
ze opnieuw vergeten deze sproeiopdracht te boeken.

Voorraadbeheer

\paragraph {} Het is gepland om in het platform niet enkel de percelen te beheren maar ook de
verschillende producten die gesproeid gaan worden. Zo kan de fruitteler de aankoop van
verschillende sproeistoffen in het platform ingeven en bij het aanmaken van een sproei
opdracht kan het platform de fruitteler een notificatie geven als een van de stoffen op raakt.
Zo kan de fruitteler tijdig een bestelling plaatsen om een nieuwe voorraad van deze stof.

vooraadbeheer

waterhuishouding

Arbeid registraties.

manuele en mechanische toepassingen (snoei,  toepassen van groei regulatoren, …)

oogst registraties.


Welke onderverdelingen worden er gemaakt
Specs per subproject wat samenvatten


