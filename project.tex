\section {project}


\subsection {Stakeholders}

\paragraph {Telers} De eindgebruikers van het systeem gaan de fruittelers zelf zijn. Zijzelf hebben weinig rechtstreekse inspraak in de ontwikkelingvan het platform. De noden van de fruittelers zullen vertegenwoordigd worden door pcfruit. Vermits een van de doelstellingen is om een groot deel van de fruittelers met deze software te laten werken, zal er vooral veel nadruk gelegd moeten worden op usability en userfriendliness. De fruittelers zijn - doordat ze de feitelijke doelgroep zijn - uitstekend om feedback over usability van terug te krijgen. Deze feedback zal via pcfruit terug tot bij ons komen in de vorm van feedback-cycles.

\paragraph {pcfruit} De software wordt ontwikkeld voor pcfruit, die heel het project financiert. pcfruit zal zelf geen eindgebruiker van het systeem zijn, maar zal als adviescentrum en productontwikkelaar op de hoogte zijn van elke stap in de ontwikkelingsprocedure. Ze worden betrokken bij de productmeetings en worden uitgenodigd bij globale meetings. pcfruit zal via showcases en testpools feedback verkrijgen van de echte eindgebruikers (de telers), en brengt deze naar voren bij de ingeplande feedback-cycles.

\paragraph {crosscomm communications NV} De implementatie en de ondersteuning van deze software zal gebeuren door Crosscom communications NV. Wij zullen de nodige analyze voorop doen. We zitten samen met pcfruit aan de tafel om de planning op te stellen en de einddoelen per fase vast te leggen. Wij zullen de nodige hardware voorzien om het platform op te draaien, en voorzien de nodige support om alles in goede banen te leiden na de deployment van de fases. Op zich zijn wij geen stakeholder van het project zelf, maar wij zullen wel de technische kant van het project op ons moeten nemen en in het oog houden. Vermits wij de uiteindelijke software schrijven, zullen wij altijd op alle meetings aanwezig moeten zijn om ons product voor te stellen aan pcfruit en mogelijke aanpassingen in te plannen of opnieuw te faseren.


\subsection {Probleembeschrijving}

\paragraph {} Een teler wil zijn oogst verkopen op een veiling waar strikte regulering is
omtrent de gewassen die verkocht worden. Zo is het bijvoorbeeld niet aanvaardbaar fruit te
verkopen waarop een te hoge dosis pesticide werd gespoten. Het bijhouden van deze
informatie gedurende het hele jaar kost veel tijd en moeite voor een teler. Zonder hulp of
advies bestaat het risico dat een onvoldoende geïnformeerde teler zijn oogst niet verkocht
krijgt op de veiling, of zelfs boetes moet betalen.


\subsection {Oplossing}

\paragraph {} pcfruit biedt een platform aan waar telers hun percelen en gewassen op
kunnen beheren. Op het platform worden de bestaande perselen gedefinieerd als ook de
gewassen die erop geteeld worden. Wanneer de teler zijn gewassen gaat besproeien kan hij
op het platorm een nieuwe sproei taak aanmaken.


\subsection {Voordelen}

\paragraph {} Bij het aanmaken van een sproei taak wordt de teler op verschillende
momenten bijgestaan door pcfruit.

\paragraph {} Een eerste voordeel is dat de informatie van de sproei taken digitaal
bijgehouden wordt.  Hierdoor kan de teler de overheid en de veiling voorzien van het
nodige papierwerk wanneer hun oogst voor verkoop wordt opgegeven. pcfruit voorziet de
teler van de nodige rapportages over alle sproei taken waarin de samenvatting van de
gebruikte stoffen te vinden is.

\paragraph {} Een tweede voordeel is dat pcfruit de teler bijstaat bij het kiezen van de
stoffen die gebruikt kunnen worden tegen specifieke kwalen. Zo kan bijvoorbeeld gezocht
worden op welke producten helpen tegen zwart-rot. Zo kan de teler steunen op de
uitgebreide databank van pcfruit om de gewassen te beschermen. Wanneer een teler een
ziekte vast stelt bij de gewassen wordt hier ook notitie van genomen zodat pcfruit kan
helpen door middel van expert advies.

\paragraph {} Een derde voordeel wordt duidelijk wanneer de teler een sproei opdracht
maakt. Tijdens het aanmaken kiest hij het perseel waarop gespoten gaat worden en het
toestel waarmee hij de sproei opdracht zal uitvoeren. Nadat de sproei opdracht is
aangemaakt kan het platform de dosering van de gebruikte producten berekenen. Zo kan de
teler zeker zijn dat hij altijd de juiste hoeveelheid stoffen gebruikt en bijgevolg ook
het juiste papierwerk kan voorleggen bij de volgende controle.


\subsection {Kadering}

\paragraph {} EVA (Eomething Vomething, Aomething), het platform aangeboden door pcfruit,
is een nieuw project dat opgestart werd ter vervanging van PitSoft. EVA pakt enkele
tekortkomingen van PitSoft aan zoals het automatisch berekenen van dosering en tank
volume. EVA zal in de toekomst verder uitgebreid worden om meer functionaliteit te bieden
aan de telers.

\subsection {Toekomstplannen}

\paragraph {} De initiele budgetering is gemaakt voor een tijdspanne van 5 jaar. Tijdens de rest van
het ontwikkelings process zullen enkele belangrijke nieuwe features geimplementeerd
worden.

Bemestingen

Gelijkaardig aan de spuit opdrachten zal pcfruit de juiste berekeningen en informatie
aanbieden voor bemestingen. Zo zullen telers geen accidentele overbesmesting meer doen en
kunnen ze de juiste rapportering aanbieden aan de overheid. Het is bijvoorbeeld
belangrijk dat de verschillende stoffen de wettelijke limieten niet overschreiden.
Kalium, fosfor en stikstof zijn wettelijk gelimiteerd. Manueel de gebruikte hoeveelheden
nitraten beheren is lastig, en tijd consuming.

Communicatie met voorlichters en adviseurs

Next gen interfaces voor mobile applicatie

Wanneer een van de werknemers door de percelen aan het wandelen is en een probleem vast
stelt met de gewassen kan hij hier een notitie van nemen op het platform. Omdat we in een
mobile omgeving werken kunnen we gebruik maken van gps locatie om automatisch het perceel
te kiezen tijdens het invullen van de waargenomen problemen.

Crowdsourcing

\paragraph {} Wanneer het aantal telers dat het platform gebruikt groeit krijgt pcfruit
meer informatie over de gewassen en de plagen die voorkomen. Dit opent de optie om data
analyse te doen en eventueel preventieve maatregels voor te stellen aan de telers.

\paragraph {} Stel dat er een aantal telers die geografisch dicht bij elkaar percelen
beheren problemen krijgen met een kever plaag. Elk van deze telers zal in het platform
deze informatie ingeven, en een sproeiactie boeken om tegen de kevers te sproeien. Een van
de andere telers zal mogelijk de plaag nog niet vast gesteld hebben. pcfruit kan door
middel van de eerder vergaarde data een suggestie sturen naar de teler die deze plaag nog
niet heeft vast gesteld om preventief tegen deze kever soort te sproeien en zo een deel
van zijn oogst redden die hij anders verloren zou hebben.

\paragraph {} Een tweede voordeel van crowdsourced data te analyseren is dat de telers ook
kunnen opgeven hoeveel oogst ze hebben gehaald uit verschillende gewassen. Omdat alle
sproeiopdrachten die de teler gedaan heeft tijdens dit jaar beschikbaar zijn, kunnen er
trends ontdekt worden in de data. Zo kan er bijvoorbeeld gemerkt worden dat een van de
telers een zeer goede oogst hebben gekregen, terwijl andere telers slechts een gemiddelde
oogst hadden. Na data analyse kan er dan geconstateerd worden dat deze teler tegen een
bepaalde kwaal geproeid heeft die de andere telers over het hoofd gezien hadden. Met deze
informatie kan er in het volgende jaar een suggestie gegeven worden aan de telers wanneer
ze opnieuw vergeten deze sproeiopdracht te boeken.

Voorraadbeheer

\paragraph {} Het is gepland om in het platform niet enkel de percelen te beheren maar ook de
verschillende producten die gesproeid gaan worden. Zo kan de teler de aankoop van
verschillende sproeistoffen in het platform ingeven en bij het aanmaken van een sproei
opdracht kan het platform de teler een notificatie geven als een van de stoffen op raakt.
Zo kan de teler tijdig een bestelling plaatsen om een nieuwe voorraad van deze stof.

vooraadbeheer

waterhuishouding

Arbeid registraties.

manuele en mechanische toepassingen (snoei,  toepassen van groei regulatoren, …)

oogst registraties.


Welke onderverdelingen worden er gemaakt
Specs per subproject wat samenvatten


