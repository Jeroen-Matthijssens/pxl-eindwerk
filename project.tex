\section {project}


\subsection {Stakeholders}

\paragraph {Telers} De eindgebruikers van het systeem gaan de fruittelers zelf zijn. Zijzelf hebben weinig rechtstreekse inspraak in de ontwikkeling van het platform. De noden van de fruittelers zullen vertegenwoordigd worden door pcfruit. Vermits een van de doelstellingen is om een groot deel van de fruittelers met deze software te laten werken, zal er vooral veel nadruk gelegd moeten worden op usability en userfriendliness. De fruittelers zijn - doordat ze de feitelijke doelgroep zijn - uitstekend om feedback over usability van te ontvangen. Deze feedback zal via pcfruit terug tot bij ons komen in de vorm van feedback-cycles.

\paragraph {Overheid} Een groot stuk van de software zal gerealiseerd worden om de communicatie naar de overheid te versimpelen. Een ander stuk zal ervoor zorgen dat de waarden en de normen die de overheid vooropstelt overzichtelijk en duidelijk weergegeven worden aan eindgebruikers. De overheid zal zelf natuurlijk niet als een actieve stakeholder plaatsnemen, maar zal net zoals de fruittelers vertegenwoordigd worden door pcfruit zelf. Mochten er wijzigingen komen ontrent wetgeving in de fruitteelt, dan zal pcfruit ons op de hoogte brengen en nieuwe feature requests aanmaken.

\paragraph {pcfruit} De software wordt ontwikkeld voor pcfruit, dat heel het project financiert. pcfruit zal zelf geen eindgebruiker van het systeem zijn, maar zal als adviescentrum en productontwikkelaar op de hoogte zijn van elke stap in de ontwikkelingsprocedure. Ze worden betrokken bij de productmeetings en worden uitgenodigd bij globale meetings. pcfruit zal via showcases en testpools feedback verkrijgen van de echte eindgebruikers (de fruittelers), en brengt deze naar voren bij de ingeplande feedback-cycles.

\paragraph {crosscomm communications NV} De implementatie en de ondersteuning van deze software zal gebeuren door Crosscom communications NV. Wij zullen de nodige analyze voorop doen. We zitten samen met pcfruit aan de tafel om de planning op te stellen en de einddoelen per fase vast te leggen. Wij zullen de nodige hardware voorzien om het platform op te draaien, en voorzien de nodige support om alles in goede banen te leiden na de deployment van de fases. Op zich zijn wij geen stakeholder van het project zelf, maar wij zullen wel de technische kant van het project op ons moeten nemen en in het oog houden. Vermits wij de uiteindelijke software schrijven, zullen wij altijd op alle meetings aanwezig moeten zijn om ons product voor te stellen aan pcfruit en mogelijke aanpassingen in te plannen of opnieuw te faseren.


\subsection {Probleembeschrijving}

\paragraph {Verkoop} Fruittelers willen op een zo simpel mogelijke manier hun fruit verkopen op een veiling of op meerdere veilingen. Veilingen willen hun kwaliteit garanderen en stellen hiervoor criteria op waar de producten - die geveild worden - aan moeten voldoen. Deze vooropgestelde regels moeten overzichtelijk en up-to-date blijven voor de fruittelers. De fruittelers moeten de criteria kunnen zien, en moeten kunnen opvolgen of ze op elk moment voldoen aan deze criteria. Een fruitteler wil niet een heel jaar fruit onderhouden, om het dan na de oogst niet te kunnen verkopen omdat ze een criterium over het hoofd gezien hebben. Een goed voorbeeld hiervoor is de dosering van pesticiden. Veilingen willen kunnen garanderen dat er ten hoogste een bepaalde hoeveelheid pesticide gebruikt is op hun producten. De fruitteler moet dan eerst en vooral al weten dat deze regel bestaat. Hij zal ook moeten weten wat de limieten van de regel zijn. Uiteindelijk zal hij dan ook een overzicht willen van de gebruikte pesticiden op zijn perceel. Op die manier blijft hij een goede kijk hebben op heel het gebeuren, en zal hij exact weten of hij op het einde van het jaar zijn oogst zal kunnen verkopen. Het is vanzelfsprekend dat deze hoeveelheid aan informatie - en het feit dat de informatie toch wel cruciaal is - een serieuze berg werk met zich meebrengt. Denk daarbij aan het feit dat fruittelers niet noodzakelijk 1 perceel hebben. Verschillende percelen onderhouden betekend ook verschillende percelen besproeien. Het zou zonde zijn als je 2 keer hetzelfde perceel zou besproeien omdat je de informatie niet correct bijgehouden hebt. Fruittelers verbouwen ook regelmatig verschillende soorten fruit. Niet alleen zal hij dan voor elk soort fruit de regels moeten onthouden, maar hij zal ook voor elk soort fruit goed moeten opletten en bijhouden hoeveel pesticide hij gebruikt. Heel deze 'boekhouding' kost enorm veel werk en vraagt enorm veel moeite om dit allemaal manueel gaan bij te houden. Het zou allemaal een heel stuk makkelijker zijn mocht er een centraal punt zijn waar alle informatie gestructureerd wordt aangereikt en waar alle persoonlijke informatie rond het produceren van fruit, per fruitteler, kan bijgehouden worden en ten alle tijde geraadpleegd kan worden. Indien heel het platform dan toch centraal staat, wil dat ook zeggen dat meerdere mensen gegevens kunnen uitwisselen indien ze dat willen. Op die manier zal je ook in teamverband kunnen samenwerken en verschillende percelen tegelijk onderhouden zonder dat er communicatieproblemen zijn.

\subsection {Oplossing}

\paragraph {} pcfruit zal een platform aanbieden waarop de fruittelers hun percelen en hun gewassen zullen kunnen beheren. Fruittelers zullen hierop hun percelen virtueel kunnen voorstellen en zullen op deze percelen allerhande virtuele acties kunnen ondernemen. Deze acties zijn eigenlijk gewoon de virtuele counterparts voor de acties die de telers op hun echte percelen toepassen. Gaat een fruitteler bemesten op een perceel, dan zal hij een virtuele actie 'bemesten' aanmaken - met een aantal specificaties zoals: type bemesting, type aanbrenging, ... etc. Deze acties zal hij dan kunnen terugvinden via eender welke internet verbinding, waarna hij de acties als uitgevoerd kan aanduiden. Op deze manier kan iemand taken aanmaken voor andere fruittelers. Op deze virtuele percelen zullen ook virtuele gewassen aangebracht worden om op die manier de representatie van de werkelijkheid zo realistisch mogelijk te houden. Na het inbrengen van de virtuele gewassen zullen een aantal van de mogelijke acties - voor het virtuele perceel met dit virtueel gewas - aangereikt worden om op die manier informatie te verschaffen aan de fruittelers. [[[Op die manier heeft het bijvoorbeeld geen zin om pesticiden te selecteren om mestkevers te bestrijden, als je een veld met bananenbomen hebt, wetende dat bananenbomen niet gevoelig zijn voor mestkever]]]. Een hele lijst van pesticideacties voor het virtuele gewas dat je op dit virtuele perceel hebt staan, zal daarentegen wel weergegeven worden om selecties en acties makkelijk aan te maken. Bij het aanmaken van acties zal er ook altijd rekening gehouden worden met alle acties die op het virtuele perceel gebeurd zijn. Op die manier zal je een notificatie gegeven worden al je al voor de 3de keer je perceel gaat bemesten, terwijl je bij de 2de bemesting eigenlijk al aan je totaal toegelaten hoeveelheid meststoffen zat. Als de fruitteler daarentegen een actie toevoegd die hem de gewassen van het perceel laat vervangen door een ander gewas, dan zal bijvoorbeeld zijn teller voor pesticiden op het huidige gewas terug op 0 gezet worden. Na al deze informatie bijhouden, weergeven en advies over de ingegeven informatie geven, kan je ook steeds afdrukken raadplegen over de huidige situatie van een perceel of een gewas. Op die manier zal de communicatie naar de verschillende veilingen en de overheid een stuk aangenamer verlopen.

\subsection {Voordelen}

\paragraph {} Bij het aanmaken van een sproei taak wordt de fruitteler op verschillende
momenten bijgestaan door pcfruit.

\paragraph {} Een eerste voordeel is dat de informatie van de sproei taken digitaal
bijgehouden wordt.  Hierdoor kan de fruitteler de overheid en de veiling voorzien van het
nodige papierwerk wanneer hun oogst voor verkoop wordt opgegeven. pcfruit voorziet de
teler van de nodige rapportages over alle sproei taken waarin de samenvatting van de
gebruikte stoffen te vinden is.

\paragraph {} Een tweede voordeel is dat pcfruit de fruitteler bijstaat bij het kiezen van de
stoffen die gebruikt kunnen worden tegen specifieke kwalen. Zo kan bijvoorbeeld gezocht
worden op welke producten helpen tegen zwart-rot. Zo kan de fruitteler steunen op de
uitgebreide databank van pcfruit om de gewassen te beschermen. Wanneer een fruitteler een
ziekte vast stelt bij de gewassen wordt hier ook notitie van genomen zodat pcfruit kan
helpen door middel van expert advies.

\paragraph {} Een derde voordeel wordt duidelijk wanneer de fruitteler een sproei opdracht
maakt. Tijdens het aanmaken kiest hij het perseel waarop gespoten gaat worden en het
toestel waarmee hij de sproei opdracht zal uitvoeren. Nadat de sproei opdracht is
aangemaakt kan het platform de dosering van de gebruikte producten berekenen. Zo kan de
teler zeker zijn dat hij altijd de juiste hoeveelheid stoffen gebruikt en bijgevolg ook
het juiste papierwerk kan voorleggen bij de volgende controle.


\subsection {Kadering}

\paragraph {} EVA (Eomething Vomething, Aomething), het platform aangeboden door pcfruit,
is een nieuw project dat opgestart werd ter vervanging van PitSoft. EVA pakt enkele
tekortkomingen van PitSoft aan zoals het automatisch berekenen van dosering en tank
volume. EVA zal in de toekomst verder uitgebreid worden om meer functionaliteit te bieden
aan de telers.

\subsection {Toekomstplannen}

\paragraph {} De initiele budgetering is gemaakt voor een tijdspanne van 5 jaar. Tijdens de rest van
het ontwikkelings process zullen enkele belangrijke nieuwe features geimplementeerd
worden.

Bemestingen

Gelijkaardig aan de spuit opdrachten zal pcfruit de juiste berekeningen en informatie
aanbieden voor bemestingen. Zo zullen fruittelers geen accidentele overbesmesting meer doen en
kunnen ze de juiste rapportering aanbieden aan de overheid. Het is bijvoorbeeld
belangrijk dat de verschillende stoffen de wettelijke limieten niet overschreiden.
Kalium, fosfor en stikstof zijn wettelijk gelimiteerd. Manueel de gebruikte hoeveelheden
nitraten beheren is lastig, en tijd consuming.

Communicatie met voorlichters en adviseurs

Next gen interfaces voor mobile applicatie

Wanneer een van de werknemers door de percelen aan het wandelen is en een probleem vast
stelt met de gewassen kan hij hier een notitie van nemen op het platform. Omdat we in een
mobile omgeving werken kunnen we gebruik maken van gps locatie om automatisch het perceel
te kiezen tijdens het invullen van de waargenomen problemen.

Crowdsourcing

\paragraph {} Wanneer het aantal fruittelers dat het platform gebruikt groeit krijgt pcfruit
meer informatie over de gewassen en de plagen die voorkomen. Dit opent de optie om data
analyse te doen en eventueel preventieve maatregels voor te stellen aan de telers.

\paragraph {} Stel dat er een aantal fruittelers die geografisch dicht bij elkaar percelen
beheren problemen krijgen met een kever plaag. Elk van deze fruittelers zal in het platform
deze informatie ingeven, en een sproeiactie boeken om tegen de kevers te sproeien. Een van
de andere fruittelers zal mogelijk de plaag nog niet vast gesteld hebben. pcfruit kan door
middel van de eerder vergaarde data een suggestie sturen naar de fruitteler die deze plaag nog
niet heeft vast gesteld om preventief tegen deze kever soort te sproeien en zo een deel
van zijn oogst redden die hij anders verloren zou hebben.

\paragraph {} Een tweede voordeel van crowdsourced data te analyseren is dat de fruittelers ook
kunnen opgeven hoeveel oogst ze hebben gehaald uit verschillende gewassen. Omdat alle
sproeiopdrachten die de fruitteler gedaan heeft tijdens dit jaar beschikbaar zijn, kunnen er
trends ontdekt worden in de data. Zo kan er bijvoorbeeld gemerkt worden dat een van de
telers een zeer goede oogst hebben gekregen, terwijl andere fruittelers slechts een gemiddelde
oogst hadden. Na data analyse kan er dan geconstateerd worden dat deze fruitteler tegen een
bepaalde kwaal geproeid heeft die de andere fruittelers over het hoofd gezien hadden. Met deze
informatie kan er in het volgende jaar een suggestie gegeven worden aan de fruittelers wanneer
ze opnieuw vergeten deze sproeiopdracht te boeken.

Voorraadbeheer

\paragraph {} Het is gepland om in het platform niet enkel de percelen te beheren maar ook de
verschillende producten die gesproeid gaan worden. Zo kan de fruitteler de aankoop van
verschillende sproeistoffen in het platform ingeven en bij het aanmaken van een sproei
opdracht kan het platform de fruitteler een notificatie geven als een van de stoffen op raakt.
Zo kan de fruitteler tijdig een bestelling plaatsen om een nieuwe voorraad van deze stof.

vooraadbeheer

waterhuishouding

Arbeid registraties.

manuele en mechanische toepassingen (snoei,  toepassen van groei regulatoren, …)

oogst registraties.


Welke onderverdelingen worden er gemaakt
Specs per subproject wat samenvatten


