\section {Situatieschets van het bedrijf}

\subsection {Wie is het bedrijf}
Crossroad Communications NV
centrum Sint-Truiden (midden in een fruit streek, geen ‘brussel of antwerpen’,
persoonlijke kleinschalige aanpak, … ).
B2B \& B2G
Communicatie tussen bedrijven en douane
jarenlange ervaring
Doel en visie van het bedrijf


\subsection {Activiteiten}


\subsubsection {B2B}

\paragraph {} communication services richting overheid (via third party partners) eerste
supportlijn voor communicatie naar de overheid custom made projects zoals FruitPC


\subsubsection {PLDA}

PaperLess Douane en Accijnzen (PLDA) is de toepassing voor het elektronisch indienen en
verwerken van aangiften. Voor het indienen van de aangiften biedt “Paperless Douane en
Accijnzen” twee mogelijkheden:
PLDA-Web: een web-applicatie, die door de administratie GRATIS ter beschikking wordt
gesteld voor het indienen van douane- en accijnsaangiften.
PLDA-EDI: een toepassing die toelaat om douane- en accijnsaangiften elektronisch in te
dienen met een EDI bericht (EDIFACT of XML) dat door uw computersysteem naar het
computersysteem van douane en accijnzen wordt verstuurd.
Crossroad Communications is gespecialiseerd in het digitaal transporteren van de PLDA-EDI
berichten. Dit zowel in XML- als EDIFACT-formaat.


\subsubsection {EMCS}

\paragraph {}De accijnstarieven in de verschillende lidstaten van de Europese Unie zijn
verschillend. Het accijnstarief van de lidstaat waarin accijnsgoederen worden verbruikt,
moet worden betaald. En wel aan de lidstaat waar de accijnsgoederen worden verbruikt.
Daarom moet vervoer van accijnsgoederen tussen de lidstaten van de Europese Unie bij de
Douane van de lidstaten bekend zijn.

\paragraph {} Om het vervoer van accijnsgoederen tussen de lidstaten met een
automatiseringssysteem te controleren ontwikkelde de Europese Unie het Excise Movement and
Control System (EMCS).  Met EMCS houdt de Douane toezicht op vervoer van accijnsgoederen
van een lidstaat naar andere lidstaten. De invoering van EMCS is gestart op 1 april 2010.
De belangrijkste verandering is de komst van het elektronisch administratief document
(e-AD). Dit vervangt het (papieren) administratief geleidedocument (AGD).


\subsubsection {Sagitta}

Alle mogelijke Nederlandse invoer aangiftes kunnen met Sagitta worden gemaakt. De
responstijden zijn meestal slechts een of enkele minuten. Er is al zo veel ervaring mee
dat de systemen inmiddels zeer betrouwbaar zijn.
Sagitta Invoer - Aangifte ten invoer
Inleiding
Om de Douane in staat te stellen alle goederen die de Gemeenschap binnenkomen te
controleren, is in de wet vastgelegd dat voor al deze goederen een aangifte moet worden
gedaan. Zowel bij het binnnenkomen in de EG, alsook wanneer deze onder een douaneregeling
worden geplaatst. Als de goederen binnenkomen en daarna weer doorgevoerd worden hoeft
slechts en summiere aangifte te worden gedaan. Worden de goederen in het vrije verkeer
gebracht, dan moet een douaneaangifte worden gedaan. Deze douaneaangifte wordt gedaan in
Sagitta-invoer. Het is niet nodig dat de eigenaar van de goederen (de importeur) zelf de
aangifte doet. Ook een derde kan daarvoor zorgdragen. Vaak is dat de douane-expediteur
die er zijn beroep van heeft gemaakt. Bij een aangifte ten invoer heft de Douane alle
rechten bij invoer en controleert zij of aan alle bepalingen op het gebied van
veiligheid, gezondheid, economie en milieu (VGEM) is voldaan. Uiteindelijk worden de
goederen “vrijgegeven” en kan de eigenaar over de goederen beschikken.
Sagitta Uitvoer - Aangifte ten uitvoer
De procedure bij uitvoer van goederen is niet veel anders dan de hiervoor beschreven
procedure bij invoer. Ook hier wordt op basis van een selectieprofiel bepaald of er moet
worden gecontroleerd. Echter, aangezien er bij uitvoer meestal slechts een gering fiscaal
belang van toepassing is, vinden er minder vaak controles plaats. De gegevens die bij de
aangifte worden aangeleverd, dienen veelal een statistisch doel (CBS) maar ook om aan te
tonen dat het nultarief voor de omzetbelasting (BTW) van toepassing is. Wanneer er wel
een fiscaal belang is (bijvoorbeeld bij landbouwrestitutie), of een niet-fiscaal belang
(in het kader van het eerder genoemde VGEM) kan de aangever rekenen op een controle. Een
voorbeeld van een VGEM-maatregel bij uitvoer zijn strategische goederen die worden
uitgevoerd.

\subsubsection {NCTS}

Wat is NCTS?
NCTS (New Computerized Transit System) is een geautomatiseerd systeem  voor het doen van
aangiften voor douanevervoer, en wordt gebruikt in alle lidstaten van de EU, alsmede in
de EFTA-landen (Noorwegen, IJsland en Zwitserland).
Hoe werkt NCTS?
De aangifte wordt electronisch aangeleverd bij de douane. Na aanvaarding krijgt de
aangifte een uniek registratienummer toegekend. Na een eventuele controle worden de
goederen vrijgegeven en kunnen deze, begeleid door het douane-document, vervoerd worden
naar de plaats van bestemming. In de tussentijd vindt er al een melding plaats vanaf het
kantoor van vertrek, naar het kantoor van bestemming. Na aankomst op het kantoor van
bestemming van de goederen, vindt er er een efficiënte controle plaats (alle relevante
gegevens zijn reeds bekend) Na controle en aanvaarding worden de goederen vrijgegeven, en
vindt er een terugkoppeling plaats richting kantoor van vertrek.

\subsubsection {XCentralPlatform en XCommAgent}

(platform) Het Xcomm Central Platform is het kernelized EDI-platform van Crossroad
Communications.
Alle meesaging, bestemd voor Customs loopt via onze XCP.
Micro-kernels zorgen voor een non-blocking, multi-threaded processing en routing van alle
EDI-messages, ongeacht de sender of recpient.

(agent) XCommAgent is een 'bridge' tussen de Customs applicatie bij de klant en het XComm
Central Platform. XcommAgent treedt op als transfer agent voor alle EDI-messages van en
naar uw applicatie.
XCommAgent is platform onafhankelijk, event driven en kan draaien als applicatie of als
daemon/service.
XCommAgent zal op een snelle, betrouwbare en geëncrypteerde wijze uw EDI-messaging
verzorgen. Dit door een event driven technologie, persistent messaging en een
SSL-verbindingsmethode.


\subsection {Bedrijfsstructuur}

5 tal mensen
Jarenlange ervaring
Flexibel door de omvang van het bedrijf (5man) zodat ze snel en gericht projecten kunnen
aannemen, bijsturen en afleveren
Steven voor support (ondersteuning van Bjorn)
Bart is de lead developer van het team
Tjo development en communicatie met de klanten
Jeroen: youtube kijken

\subsection {Persoonlijke Motivatie}

\subsubsection {Bedrijf keuze}

Saas … 1 product, verschillende klanten voor dit product.
productkennis en productvisie zijn belangrijk en vormen een rode draad door het onderhoud
en de ontwikkeling van het product.
je hebt een eigen codebase en je onderhoudt deze ook

\subsubsection {Project keuze}

er werd een nieuw project binnengehaald dat mooi in delen onderverdeeld werd. Dit
onderdeel staat volledig los waardoor het perfect afgeleverd kan worden in de periode die
samengaat met de stageperiode.
Het grote project, waar mijn stage een klein onderdeel van is, staat nog in zijn
ontwerpfases waardoor ik met dit component de volledige flow kan meemaken. Zowel qua
research, productdesign, finetunen van het product, als de communicatie met de klant
meemaken en in development cycli meewerken met feedback van de klant. Hier komt dan ook
bij dat de klant waarmee we werken - pcfruit - zich ook in Sint Truiden bevindt, waardoor
de interactie sneller en duidelijker kan.

\subsubproject {Verwachtingen}
Het project zal groot genoeg zijn om er met een team aan te werken. Samenwerken met
andere mensen van het team zou een aantal dingen in gang steken. Samen werken impliceert
samen communiceren, dus er zouden zeker nieuwe methodologieen gebruiken worden zoals
scrum. Sprints zou ook een concept zijn waar we focus op gingen leggen. Geregeld zouden
er retrospective meetings moeten plaatsvinden om het hele team op dezelfde golflengte te
houden. Dit waren vooral de dingen die ik zag aankomen en die ik zou verwachten bij het
creeeren van een full stack applicatie die toch wel een serieuze userbase zou moeten
aanraken.
Qua technologieen wist ik dat er veel nieuws tevoorschijn zou komen, maar ik had hier
niet dadelijk specifieke verwachtingen voor. Ik verwachtte hier zowel nieuwe technieken
te gebruiken, als technieken die ik al kende opnieuw te gebruiken en meer specifiek en
meer in de praktijk ging gebruiken.
Het is ook het eerste project, waar ik aan meegewerkt heb, dat effectief opgeleverd moest
worden aan een klant. Hiervoor werkte ik vooral mee aan SAAS projecten, en had dus de
interactie niet met de klant. Deze interactie, samen met de voorgenoemde methodologieen
zouden een interessante vernieuwing zijn.
